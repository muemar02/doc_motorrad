 \chapter{Koordinatensysteme}\label{ch:kos}
 
 \section{Rechtssystem}\label{sec:kos_rechtssys}
  Gegeben sei ein euklidisches Koordinatensystem $\KOS{I}\in \R^{3}$ mit den Basisvektoren $\vect{e}_1,\vect{e}_2,\vect{e}_3$. F\"ur die Basisvektoren gelte: \begin{align}
\skalar{\vect{e}_i}{\vect{e}_j}&=
\begin{cases}
1, \text{ f\"ur } i=j \\
0, \text{ f\"ur } i\neq j \end{cases} \label{gl:kosEvSenkr}
\intertext{und weiterhin} 
\vect{e}_1 \times \vect{e}_2 &= \vect{e}_3 \label{gl:kosEvRechtssys}
\end{align}
Die Basisvektoren von $\KOS{I}$ beschreiben damit ein orthonormales Rechtssystem (siehe beispielsweise \cite[S. 80]{Papula2014}). Im Folgenden wird, wenn nicht explizit anderweitig angegeben, davon ausgegangen, dass alle verwendeten Koordinatensysteme diese Eigenschaften erf\"ullen.  
  
  \section{Nat\"urliche Koordinaten}\label{sec:kos_natKoord}
  engl. natural coordinates wie ist der richtige deutsche Begriff?
  
  \section{Homogene Koordinatensysteme}\label{sec:kos_homKoord}
  Translation eines Vektors $\vect{v}=\transp{\of{x, y, z}}$ mit einem Vektor $\vect{q}=\transp{\of{a, b, c}}$:\begin{align*}
  \vect{v}' &=\begin{bmatrix}
  1 & 0 & 0& a\\
  0 & 1 & 0& b\\
  0 & 0 & 1& c\\
  0 & 0 & 0& 1\\
  \end{bmatrix} \begin{bmatrix}
  x \\ y \\ z \\1
  \end{bmatrix} = \begin{bmatrix}
  x+a \\ y+b \\ z+c \\ 1
  \end{bmatrix}.
  \intertext{statt}
  \vect{v}'&= \begin{bmatrix}
  x \\ y \\ z
  \end{bmatrix} + \begin{bmatrix}
  a \\ b\\c
  \end{bmatrix} = \begin{bmatrix}
  x+a \\ y+b \\ z+c
  \end{bmatrix}.
  \end{align*}
  Eine Rotationsmatrix $\matr{R}$ erf\"ullt stets die Bedingung: \begin{align*}
  \matr{R}\transp{\matr{R}}&= E
  \end{align*} Eine Transformationsmatrix $\matr{T}$, welche sich aus Rotation $\matr{R}$ und Translation $\vect{q}$ zusammensetzt, wird in homogenen Koordinaten beschrieben mit: \begin{align*}
  \matr{T}&= \begin{bmatrix}
  \matr{R} & \vect{q}\\ 
  \vect{0} & 1
  \end{bmatrix} = \begin{bmatrix}
  R_{11} & R_{12} & R_{13} & q_{1} \\
  R_{21} & R_{22} & R_{23} & q_{2} \\
  R_{31} & R_{32} & R_{33} & q_{3} \\
  0&0&0&1
  \end{bmatrix}
  \end{align*} Die Inverse dieser Transformationsmatrix berechnet sich unter Beachtung der Orthogonalit\"atseigenschaft von $\matr{R}$ zu \begin{align*}
  \inv{\matr{T}}&= \begin{bmatrix}
  \transp{\matr{R}} & - \transp{\matr{R}}\vect{q}\\ 
  \vect{0} & 1
  \end{bmatrix}
  \end{align*} Betrachtet man hingegen eine Transformationsmatrix im $\R^{3}$, welche eine Rotation beinhaltet, so wird die inverse dieser Transformation nach folgenden Regeln berechnet: \begin{align*}
  \matr{T}&= \matr{R} \\
  \inv{T}&=\inv{\matr{R}} = \transp{\matr{R}}
  \end{align*}
  \section{Koordinatentransformation} \label{sec:kos_transf}
  Gegeben seien ein inertiales Koordinatensystem $\KOS{I}$ und zwei, um eine beliebige Achse in Relation zu $\KOS{I}$ gedrehte, Koordinatensysteme $\KOS{B},\KOS{C}$. 
  \paragraph*{Rotationen}
  Es sei ein Punkt $q_b= \transp{\left( x_b, y_b, z_b \right) }$ im Koordinatensystem $\KOS{B}$ gegeben. Werden die Koordinatenachsen von $\KOS{B}$ durch die Einheitsvektoren $\vect{e}_{1b}, \vect{e}_{2b}, \vect{e}_{3b}$ im Inertialsystem $\KOS{I}$ beschrieben, so kann der Punkt von $\KOS{B}$ nach $\KOS{I}$ durch eine Transformation \"uberf\"uhrt werden: \begin{align*}
  q_i&= \begin{pmatrix}
  \vect{e}_{1b} & \vect{e}_{2b} & \vect{e}_{3b}
  \end{pmatrix} \begin{pmatrix}
  x_b\\ y_b\\ z_b
  \end{pmatrix} = \matr{R}_{ib} q_b
\end{align*}
  Der Index der Transformationsmatrix ist so zu verstehen, dass der erste Buchstabe das Zielsystem und der zweite Buchstabe das Ursprungssystem der Transformationsmatrix angibt. \hfill \newline  
  Analog zur Transformation eines Punktes kann auch ein Vektor $\vect{v}_b=q_b - p_b$, welcher im System $\KOS{B}$ definiert wurde, in das System $\KOS{I}$ transformiert werden: \begin{align*}
  \vect{v}_i &= \matr{R}_{ib} \vect{v}_b = \matr{R}_{ib} q_b - \matr{R}_{ib} p_b = q_i - p_i.
  \end{align*}
  Weiterhin k\"onnen Transformationen aneinandergereiht werden. Beschreibt die Transformationsmatrix $\matr{R}_{bc}$ die Verdrehung von $\KOS{C}$ relativ zu $\KOS{B}$, so erh\"alt man die Transformationsmatrix von $\KOS{C}$ nach $\KOS{I}$ durch eine Kombination der Transformation vom System $\KOS{C}$ in das System $\KOS{B}$ mit der Transformation vom System $\KOS{B}$ in das System $\KOS{I}$. Die Kombination erfolgt dabei durch linksseitige Matrixmultiplikation der jeweiligen Transformationsmatrizen in der angegebenen Reihenfolge. \begin{align*}
  \matr{R}_{ic}&= \matr{R}_{ib} \matr{R}_{bc}
  \end{align*}
  \paragraph*{Rotation und Translation}
  
  \section{Rotationsmatrizen}\label{sec:kos_rotmatr}
  Gegeben sei ein Koordinatensystem $\KOS{K}$, welches um eine Achse $l$ relativ zu einem inertialen Koordinatensystem $\KOS{I}$ gedreht wurde. Die Orientierung dieser Drehachse sei durch einen Vektor $\vect{l}$ mit Einheitsl\"ange beschrieben. Die Achsen von $\KOS{K}$ relativ zu $\KOS{I}$ seien gegeben durch die Vektoren $\vect{u}, \vect{w}, \vect{v} \in \R^{3}$. Die drei Spaltenvektoren werden horizontal zu einer Matrix \begin{align}
  \matr{R}&=\begin{bmatrix}
  \vect{u} & \vect{w} & \vect{v}
  \end{bmatrix} = \begin{pmatrix}
  u_x & w_x & v_x \\ u_y & w_y & v_y \\ u_z & w_z & v_z
  \end{pmatrix} \label{gl:rotMatrDef}
  \end{align}
  zusammengefasst. Die Matrix $\matr{R}$ wird als Rotationsmatrix bezeichnet. 
    \subsection{Eigenschaften von Rotationsmatrizen}\label{ssec:kos_rotmatr_eigenschaften}
    Die Spalten der Rotationsmatrix $\matr{R}\in \R^{3\times 3}$ seien die Vektoren $\vect{u}, \vect{w}, \vect{v} \in \R^{3}$. Da diese ein Koordinatensystem aufspannen haben sie die in \eqnref{gl:kosEvSenkr} und \eqnref{gl:kosEvRechtssys} definierten Eigenschaften. Aus \eqnref{gl:kosEvSenkr} folgt f\"ur die Matrix $\matr{R}$ \begin{align}
    \matr{R}\transp{\matr{R}} &= \begin{pmatrix}
    \vect{u} & \vect{w}& \vect{v} 
\end{pmatrix} \begin{pmatrix}
\vect{u}\\ \vect{w}\\ \vect{v}  
\end{pmatrix}     \transp{\matr{R}}\matr{R} = \matr{I} \label{gl:rotMatrTransp} 
    \intertext{und mit Hilfe der Regeln der linearen Algebra \cite[S. 100]{Papula2014} und \eqnref{gl:kosEvSenkr}}
    \det{\matr{R}} &= \skalar{\vect{u}}{\vect{w}\times \vect{v}}= \skalar{\vect{u}}{\vect{u}} =  1 \label{gl:rotMatrDet}
\end{align} Die Menge der orthogonalen $3 \times 3$ Matrizen mit der Determinante eins wird als $\set{SO}\of{3}$ bezeichnet \cite{Murray1994}. Allgemein wird definiert: \begin{align}
\set{SO}\of{n} = \left \lbrace \matr{R}\in \R^{n\times n}: \matr{R}\transp{\matr{R}}=\matr{I}, \det{\matr{R}}=+1 \right \rbrace. \label{gl:mengeSoDef}
\end{align} Die Menge $\set{SO}\of{3}\subset \R^{3\times 3}$ bildet mit der Abbildungsvorschrift \textit{Matrixmultiplikation} eine Gruppe entsprechend den im Abschnitt \ref{sec:mathGrundl_gruppen} geforderten Regeln. Die geforderten Eigenschaften werden wie folgt erf\"ullt:
\begin{itemize}
\item Die Verkn\"upfung ist abgeschlossen. F\"ur $\matr{R}_1, \matr{R}_2 \in \set{SO}\of{3}$ gilt auch $\matr{R}_1 \matr{R}_2 \in \set{SO}\of{3}$, da \begin{align}
\matr{R}_1 \matr{R}_2 \transp{\of{\matr{R}_1 \matr{R}_2}} &= \matr{R}_1 \matr{R}_2 \transp{\matr{R}_2} \transp{\matr{R}_1} = \matr{R}_1 \transp{\matr{R}_1} = \matr{I} \\
\det\of{\matr{R}_1 \matr{R}_2} &= \det\of{\matr{R}_1} \det\of{\matr{R}_2} = +1
\end{align} gilt.
\item Die Gruppe $\set{SO}\of{3}$ ist assoziativ \begin{align}
\intertext{Aus der Assoziativit\"at der Matrixmultiplikation (Beweis siehe z. B. \cite[s. 93]{Bosch2014}) folgt}
\left( \matr{R}_1 \matr{R}_2\right) \matr{R}_3 &= \matr{R}_1 \left(\matr{R}_2 \matr{R}_3 \right) \label{gl:mengeSoGruppenAssoz}
\end{align}
\item Die Einheitsmatrix ist das neutrale Element \begin{align}
\matr{I} \matr{R} = \matr{R} \matr{I} = \matr{R} \quad \forall { } \matr{R} \in \set{SO}\of{3} \label{gl:mengeSoGruppenNeutrElem}
\intertext{mit} 
\matr{I}&=\begin{pmatrix}
1 & 0 & 0\\ 0 & 1 & 0 \\ 0 & 0 & 1
\end{pmatrix} \nonumber
\end{align}
\item Aus \eqnref{gl:rotMatrTransp} folgt, dass $\transp{\matr{R}}\in \set{SO}\of{3}$ das inverse Element von $\matr{R} \in \set{SO}\of{3}$ ist.
\end{itemize}
\begin{rem} Die Lage eines Starrk\"orpers, welcher sich frei im Raum drehen kann, kann zu jedem Zeitpunkt durch eine eindeutige Rotationsmatrix $\matr{R}\in \set{SO}\of{3}$ beschrieben werden. Die Menge der Rotationsmatrizen $\set{SO}\of{3}$ wird daher als der Konfigurationsraum des Systems bezeichnet. Eine Trajektorie des Systems wird durch die Kurve $\matr{R}\of{t} \in \set{SO}\of{3}$ f\"ur $t\in [0,T]$ abgebildet. Weiterhin dient die Matrix $\matr{R}$ zur Transformation von Punkten von einem k\"orperfesten, um eine beliebige Achse gedrehten, Koordinatensystem in ein Inertialsystem. Diese Funktion ist in Abschnitt \ref{sec:kos_transf} genauer dargelegt.
\end{rem}