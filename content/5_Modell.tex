\chapter{Motorradmodell}\label{ch:modell}
Die Modellierung von Motorr\"adern ist Gegenstand aktueller Forschung. Insbesondere die Arbeiten von R. S. Sharp \cite{Sharp1971}, \cite{SHARP1985}, \cite{Sharp2001} sowie die Arbeiten von V. Cossalter und R. Lot \cite{Cossalter2002}, \cite{Cossalter2010} werden vielfach zitiert. Eine \"Ubersicht \"uber die historische Entwicklung von Motorradmodellen und zahlreiche Quellen sind in \cite{Limebeer2006} zu finden. Au\ss{}erdem liefert \cite{Schwab2013} eine umfangreiche Quellen\"ubersicht zur Modellierung und dem Fahrverhalten von Einspurfahrzeugen. Weitere aktuelle Modelle sind unter anderem in \cite{Kanoh2007}, \cite{Nehaoua2013}, dem umfangreichen Buch von M. Tanelli \cite{MaraTanelli2014a} und \cite{Leonelli2015} dargestellt. Zus\"atzlich sei noch das Buch von J. Stoffregen \cite{Stoffregen2012} genannt, welches Grundlagen und Konzepte der Motorradtechnik beleuchtet.  \hfill \newline
Es existieren offensichtlich zahlreiche Modelle, welche die Bewegung von Motorr\"adern beschreiben. Deren Genauigkeit unterscheidet sich teilweise stark. Besonders der Einfluss der Bewegung des Fahrers wird oft nicht oder nur rudiment\"ar modelliert, obwohl die Masse das Fahrers im Vergleich zur Fahrzeugmasse signifikant ist. Weiterhin haben die Anzahl an Teilk\"orpern, in welche ein Motorrad zerlegt wird, die Beachtung von Federn und D\"ampfern und die Modellierung der Reifen gro\ss{}en Einfluss auf die Modellgenauigkeit. Ein besonders Augenmerk liegt dabei auf dem genauen Kontaktpunkt zwischen Reifen und Stra\ss{}e und den dort wirkenden Kr\"aften. Diese sind unter anderem abh\"angig von der Schr\"aglage des Motorrades, der Reifengeometrie und der Reifenverformung, welche durch Gravitationskr\"afte und Antriebs- beziehungsweise Bremsmomente entsteht. F\"ur die Beschreibung der Reifenkr\"afte ist die Entwicklung der \textit{Magic Formula} durch H. B. Pacejka und E. Bakker \cite{Pacejka1992} das meist verwendete Standardwerkzeug. Dieser weit verbreitete Ansatz wurde vielfach weiterentwickelt und speziell auf Motorradreifen angepasst. Das Buch von P. Flores et. al. \cite{Gent2006} greift diesen Ansatz auf und erl\"autert dessen Anwendung.  Die Arbeiten \cite{Besselink2010}, \cite{Pacejka2012}, \cite{Redrouthu2014} und \cite{Lot2004} sind eine kleine Auswahl aktueller Arbeiten zu diesem Thema. \hfill \newline

Das von V. Cossalter und R. Lot in \cite{Cossalter2002} vorgestellte Motorradmodell hebt sich in sofern von der Masse der bekannten Motorradmodelle ab, als es laut Aussage der Autoren das einzige Modell ist, welches trotz hoher Modellgenauigkeit in Echtzeit berechnet werden kann. Speziell dieses Modell ist daher wahrscheinlich gut geeignet, um als Basis f\"ur Algorithmen zu fungieren, welche Motorradkenndaten, wie beispielsweise die w\"ahrend der Fahrt wirkenden Reifenkr\"afte, in Echtzeit ben\"otigen. Solche Algorithmen k\"onnten zum Beispiel die wahrscheinlichsten Trajektorien ermitteln, welche ein Motorrad in unmittelbarer Zukunft befahren wird, um so eventuelle Kollisionen oder anderen gef\"ahrliche Situationen vorherzusehen. Bei Erwartung einer gef\"ahrlichen Situationen k\"onnten dann entsprechende Gegenma\ss{}nahmen eingeleitet werden, um Sch\"aden beziehungsweise Verletzungen f\"ur Maschine und Fahrer zu vermeiden oder wenigstens abzuschw\"achen. \hfill \newline
Im Folgenden werden einige Gleichungen des von V. Cossalter und R. Lot im Jahr 2002 in \cite{Cossalter2002} ver\"offentlichten Motorradmodells beispielhaft untersucht, da sie eine direkte Anwendung der in den Kapiteln \ref{ch:kos} und \ref{ch:mech} hergeleiteten Beziehungen zur Beschreibung der Bewegung von K\"orpern darstellen. 

\subsection{Hier kommen jetzt noch ein paar Gleichungen hin anhand derer die Anwendung von Kap. 3 und 4 gezeigt werden}