\chapter{Zusammenfassung und Ausblick}
In der vorliegenden Arbeit wurden die mathematischen Strukturen erl\"autert, welche zur Beschreibung des dreidimensionalen Raumes mit Vektoren notwendig sind. Die Rechtssysteme wurden eingef\"uhrt, mit deren Hilfe mechanische Problemstellungen modelliert und gel\"ost werden k\"onnen. Die \"ubliche Darstellung von Vektoren als Tripel wurde um eine vierte Komponente erweitert. Es wurde gezeigt, dass die so erhaltene Darstellung die Beschreibung der Bewegung eines Starrk\"orpers mittels linearer Abbildungen erm\"oglicht. Au\ss{}erdem wurde gezeigt, wie mit Hilfe homogener Koordinaten komplexe Transformationen kompakt darstellbar sind. Die Parametrierung von Mehrk\"orpersystemen wurde anhand der Methode der nat\"urlichen Koordinaten erl\"autert. Dar\"uber hinaus wurden grundlegende kinematische Beziehungen unter Beachtung der Verwendung homogener Koordinaten hergeleitet. Insbesondere deren kompakte Formulierung war ein wichtiger Teil dieser Arbeit. Anhand eines Mehrk\"orpermodells f\"ur ein Motorrad wurden die aufgezeigten Verfahren beispielhaft angewandt. Insbesondere die simple Formulierung von Zwangsbedingungen wurde deutlich gemacht. Es zeigte sich, dass Mehrk\"orpersysteme bei Parametrierung mit nat\"urlichen Koordinaten und unter Verwendung einer homogenen Koordinatendarstellung mit dem Ansatz von Lagrange \mbox{2. Art} als System von \acp{dae} darstellbar sind. Die bei der L\"osung von \acp{dae} auftretenden Probleme wurden aufgezeigt und m\"ogliche L\"osungen referenziert. \hfill \newline
F\"ur k\"unftige Arbeiten bleibt vor allem die L\"osung der \acp{dae}, welche sich aus der Modellierung von mechanischen Systemen ergeben. Insbesondere die Berechnung konsistenter Anfangswerte und die durch Indexreduktion notwendige Stabilisierung der Gleichungen sind Problemstellungen ohne allgemeine L\"osungen.   