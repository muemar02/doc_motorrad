
\chapter*{Bibliografischer Nachweis}
\thispagestyle{empty}
%\pagestyle{empty}
Marius M\"uller\\[2ex]
\textbf{Modellierung technischer Systeme mit Hilfe homogener Koordinaten am Beispiel eines Motorradmodells} \newline
Studienarbeit: 73 Seiten, 5 Abbildungen, 67 Literaturangaben\\
15.01.2017 \\
Technische Universit\"at Dresden \\
Fakult\"at Verkehrswissenschaften \glqq Friedrich List\grqq \\
Institut für Verkehrstelematik\\[2ex]
Autorenreferat:\\
Technische Systeme k\"onnen auf verschiedenste Arten und zu verschiedensten Zwecken modelliert werden. Bei mechanischen Systemen ist meist die Bewegung von Interesse. Die Aufteilung in mehrere Starrk\"orper ist ein \"ublicher Ansatz der Modellbildung. Ein derart erhaltenes Mehrk\"orpersystem muss so durch Parameter beschrieben werden, dass die Lage aller Teilk\"orper bestimmt, und damit die Bewegung des Gesamtsystems beschrieben werden kann. Die Wahl von Parametern nach der Methode der nat\"urlichen Koordinaten wird erl\"autert. Die Bewegungsbeschreibung geschieht \"ublicherweise mit Vektoren, welche als Elemente des $\set{R}^{3}$ aufgefasst werden. Die mathematischen Grundlagen dieser Beschreibungsform werden dargelegt. Die Repr\"asentation des Ortes als Tripel erlaubt keine Darstellung der Starrk\"orperbewegung in Form einer linearen Abbildung. Dieses Problem wird durch die Einf\"uhrung homogener Koordinaten gel\"ost. Mit Hinblick auf diese Variante der Koordinatendarstellung werden Grundprinzipien der Kinematik hergeleitet und es wird auf die L\"osung von Bewegungsgleichungen eingegangen. Die Verwendbarkeit dieser Darstellung wird anhand ausgew\"ahlter Modellgleichungen, mit denen ein Motorrad beschrieben werden kann, gezeigt.

