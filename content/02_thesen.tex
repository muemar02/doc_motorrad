\chapter*{Thesen zur Studienarbeit}
\begin{enumerate}
\item Die Elemente des Vektorraumes $\set{R}^{3}$ \"uber dem K\"orper der reellen Zahlen sind ein geeignetes Mittel, um Punkte im Anschauungsraum zu beschreiben.
\item Wird ein Starrk\"orper mit Hilfe von Vektoren in Form von Tripeln beschrieben, so kann die Translation eines Starrk\"orpers nicht durch eine lineare Abbildung beschrieben werden. 
\item Die Translation eines Starrk\"orpers kann durch eine lineare Abbildung dargestellt werden, wenn die Darstellung der Bewegung durch Tripel geeignet erweitert wird.
\item Homogene Koordinaten erm\"oglichen eine simple Darstellung komplexer Transformationen von Punkten eines Starrk\"orpers.
\item Die Parametrierung eines Mehrk\"orpersystems mit nat\"urlichen Koordinaten ist m\"oglich. 
\item Die Parametrierung eines Mehrk\"orpersystems mit nat\"urlichen Koordinaten erm\"oglicht eine besonders simple Formulierung von Zwangsbedingungen.  
\end{enumerate}