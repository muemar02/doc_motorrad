\chapter{Vorwort}
Problemstellungen der Kinetik lassen sich in geeigneter Weise im dreidimensionalen Raum formulieren. Dabei dient dieser zur Beschreibung der Lage und der Lage\"anderung des zu analysierenden Systems. \newline
Um den dreidimensionalen Raum selbst beschreiben zu k\"onnen wurden zahlreiche mathematische Konzepte entwickelt. Im Folgenden soll ein grundlegender \"Uberblick \"uber diese Konzepte gegeben werden. Sie sind dem Fachgebiet der Linearen Algebra zuzuordnen. Ihre Darstellung ist dem Lehrbuch \cite{Bosch2014} entnommen. Anhand eines Beispiels werden einige wichtige Begriffe eingef\"uhrt. Diese werden in den folgenden Abschnitten detailliert erkl\"art. Anschlie\ss{}end werden die zur Modellierung verwendeten kinematischen Konzepte erl\"autert und das Modell hergeleitet.