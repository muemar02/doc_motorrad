\chapter{Vorwort}
Problemstellungen der Kinetik lassen sich in geeigneter Weise im dreidimensionalen Raum formulieren. Dabei dient dieser zur Beschreibung der Lage und der Lage\"anderung des zu analysierenden Systems. \newline
Um den dreidimensionalen Raum selbst beschreiben zu k\"onnen wurden zahlreiche mathematische Konzepte entwickelt. Im Folgenden soll ein grundlegender \"Uberblick \"uber diese Konzepte gegeben werden, welche dem Fachgebiet der Linearen Algebra zuzuordnen sind. Ihre Darstellung ist dem Lehrbuch von S. Bosch \cite{Bosch2014} entnommen. Anhand eines Beispiels werden einige wichtige Begriffe eingef\"uhrt. Diese werden in den folgenden Abschnitten detailliert erkl\"art. Anschlie\ss{}end werden die zur Modellierung verwendeten kinematischen Konzepte erl\"autert. Insbesondere werden die Homogenen Koordinaten als eine Darstellungsvariante von Punkten und deren Bewegung im dreidimensionalen Raum eingef\"uhrt. Weiterhin wird das Konzept der nat\"urlichen Koordinaten erkl\"art, mit Hilfe dessen ein System von Starrk\"orpern parametriert werden kann. Die Bewegung von Starrk\"orpern wird speziell unter Verwendung dieser zwei Konzepte untersucht. Anhand ausgew\"ahlter Gleichungen aus der Arbeit von V. Cossalter und R. Lot \cite{Cossalter2002} werden die eingef\"uhrten Konzepte exemplarisch dargelegt.