\chapter{Stand der Technik}\label{ch:SdT}
\section{Mathematische Grundlagen}\label{sec:SdT_mathGrundl}
  \subsection{Koordinatensysteme}\label{ssec:SdT_mathGrundl_kos}
  Gegeben sei ein euklidisches Koordinatensystem $\KOS{I}\in R^{3}$ mit den Basisvektoren $\vect{e}_1,\vect{e}_2,\vect{e}_3$. F\"ur die Basisvektoren gelte: \begin{align}
\skalar{\vect{e}_i}{\vect{e}_j}&=
\begin{cases}
1, \text{ f\"ur } i=j \\
0, \text{ f\"ur } i\neq j \end{cases} \label{gl:kosEvSenkr}
\intertext{und weiterhin} 
\vect{e}_1 \times \vect{e}_2 &= \vect{e}_3 \label{gl:kosEvRechtssys}
\end{align}
Die Basisvektoren von $\KOS{I}$ beschreiben damit ein orthonormales Rechtssystem (siehe beispielsweise \cite[S. 80]{Papula2014}). Im Folgenden wird, wenn nicht explizit anderweitig angegeben, davon ausgegangen, dass alle verwendeten Koordinatensysteme diese Eigenschaften erf\"ullen.  
  \subsection{Punkte und Vektoren}\label{ssec:SdT_mathGrundl_punkteVektoren}
  Zur Beschreibung eines Punktes wird ein Koordinatensystem ben\"otigt. Ein Punkt wird eindeutig durch seine \textit{Position} relativ zu diesem Koordinatensystem beschrieben. Als Koordinatensystem wird das Koordinatensystem $\KOS{I}$ aus \ref{ssec:SdT_mathGrundl_kos} verwendet. Die Position eines Punktes $p$ kann dann wie folgt beschrieben werden: \begin{align*}
  p &=  \left(x | y | z\right)\in \R^{3} = x  \vect{e}_1 + y \vect{e}_2 + z \vect{e}_3 
  \end{align*} Die Position von $p$ ist damit relativ zu $\KOS{I}$ eindeutig beschrieben. \newline
  Ein Vektor $\vect{a}\in \R^{3}$ hat im Gegensatz zum Punkt eine \textit{Richtung} und einen \textit{Betrag} beziehungsweise eine L\"ange. Die Begriffe Betrag und L\"ange werden im Folgenden synonym verwendet. Mit beiden Eigenschaften ist die euklidische Norm des Vektors $\vect{a}$ gemeint. Folgende Darstellung wird f\"ur den Betrag eines Vektors $\vect{a}$ verwendet:
  \begin{align*}
  \abs{\vect{a}}:= \norm{\vect{a}}
  \end{align*}
Eine Vektor kann frei im Raum verschoben werden, so lange seine Richtung und sein Betrag konstant bleiben. Man spricht daher auch von freien Vektoren. F\"ur die Darstellung von Vektoren gibt es verschiedene M\"oglichkeiten. Eine Variante ist die Angabe von Anfangspunkt $p$ und Endpunkt $q$
\begin{align*}
\vect{a}&= q - p = \vect{qp}.
\end{align*}
Da Vektoren frei verschiebbar sind, ist die Wahl der Anfangs- und Endpunkte jedoch nicht eindeutig. Es gibt daher andere Punkte $r, s$, f\"ur die gilt:
\begin{align*}
\vect{a}&= q - p = r - s 
\intertext{mit}
q&\neq r \text{ und } p\neq s
\end{align*}
Eine weitere Darstellung ist die Komponentenschreibweise bez\"uglich eines Koordinatensystems $\KOS{I}$, welche durch Projektion auf die Basisvektoren von $\KOS{I}$ gegeben ist:
\begin{align*}
\vect{a}&=  \vect{a}_x + \vect{a}_y + \vect{a}_z =  x \vect{e}_1 + y \vect{e}_2 + z \vect{e}_3 = 
\begin{pmatrix} x \\ y \\ z 
\end{pmatrix} 
\intertext{damit ergibt sich der Betrag eines Vektor zu}
\abs{\vect{a}}&= \sqrt{x^2 + y^2 + z^2} = a
\end{align*}  

Folgende Eigenschaften gelten f\"ur Vektoren $\vect{a}=\begin{pmatrix} x \\ y \\ z \end{pmatrix}, \vect{b}=\begin{pmatrix} x' \\ y' \\ z' \end{pmatrix}$ (nach \cite{Papula2014}):

\begin{itemize}
	\item Vektoren sind \textit{gleich}, wenn sie in Richtung und Betrag \"ubereinstimmen \begin{align*}
	\vect{a}=\vect{b} &\Leftrightarrow \abs{a}=\abs{b} \wedge \vect{a} \uparrow \uparrow \vect{b}
	\end{align*}
	
	\item \textbf{Addition}/ Subtraktion von Vektoren erfolgt durch Addition/ Subtaktion der Komponenten
	\begin{align*}
	\vect{a}\pm\vect{b}&=\begin{pmatrix} x \\ y \\ z \end{pmatrix} \pm \begin{pmatrix} x' \\ y' \\ z' \end{pmatrix} = \begin{pmatrix} x\pm x' \\ y\pm y' \\ z\pm z' \end{pmatrix} 
	\end{align*}
	Es gelten die Rechenregeln:
	  \begin{itemize}
	  \item Kommutativgesetz: \begin{align*}
	  \vect{a}\pm\vect{b} &= \pm\vect{b} + \vect{a}
	  \end{align*}
	  \item Assoziativgesetz: \begin{align*}
	  \vect{a}+ \left( \vect{b}+\vect{c} \right) &= \left( \vect{a}+\vect{b} \right) + \vect{c}
	  \end{align*}
	  \end{itemize}
	  
	\item Multiplikation mit einem Skalar $\lambda, \mu \in \R$ erfolgt durch Multiplikation aller Komponenten mit dem Skalar \begin{align*}
	\lambda\cdot \vect{a}&= \begin{pmatrix} \lambda\cdot x \\ \lambda\cdot y \\ \lambda\cdot z \end{pmatrix}
	\end{align*}
	Es gelten die Rechenregeln:
	  \begin{itemize}
	  \item Distributivgesetz: \begin{align*}
	  \lambda \left( \vect{a}+\vect{b}\right) &= \lambda\vect{a} + \lambda\vect{b}
	  \end{align*}
	  \item weitere Regeln: \begin{align*}
	  \left(\lambda + \mu \right) \vect{a}&= \lambda\vect{a} + \mu\vect{a} \\
	  \left(\lambda \mu \right) \vect{a}&= \lambda \left( \mu  \vect{a} \right) =  \mu \left(\lambda  \vect{a} \right)\\ 
	  \abs{\lambda \vect{a}}&= \abs{\lambda}\abs{\vect{a}}
	  \end{align*}
	  \end{itemize}
	
	\item Das \textbf{Skalarprodukt} \acs{skalarProd} zweier Vektoren ist das Produkt der Betr\"age und dem Kosinus des von den Vektoren eingeschlossenen Winkels $\varphi$ \begin{align*}
	\skalar{\vect{a}}{\vect{b}}&= \abs{a}\abs{b}\cos{\varphi} = \left( x \vect{e}_1 + y \vect{e}_2 + z \vect{e}_3\right) \left( x' \vect{e}_1 + y' \vect{e}_2 + z' \vect{e}_3 \right)
	\end{align*}
	Es gelten die Rechenregeln:
	  \begin{itemize}
	  \item Kommutativgesetz: \begin{align*}
	  \skalar{\vect{a}}{\vect{b}} &= \skalar{\vect{b}}{\vect{a}}
	  \end{align*}
	  \item Distributivgesetz: \begin{align*}
	  \skalar{\vect{a}}{\vect{b}+\vect{c}} &= \skalar{\vect{a}}{\vect{b}} + \skalar{\vect{a}}{\vect{c}}
	  \end{align*}
	  \item weitere Regeln: \begin{align*}
	  \lambda \skalar{\vect{a}}{\vect{b}} &= \skalar{\lambda \vect{a}}{\vect{b}} = \skalar{ \vect{a}}{\lambda\vect{b}}
	  \end{align*}
	  \end{itemize}
	  \begin{rem}[Orthogonale Vektoren] Verschwindet das Skalarprodukt zweier von Null verschiedenen Vektoren, so stehen diese senkrecht aufeinander. \begin{align*}
	  \skalar{\vect{a}}{\vect{b}}&=0 \Leftrightarrow \vect{a} \perp  \vect{b}
    \end{align*}	   
    \end{rem}
    \begin{rem}[Winkel zwischen Vektoren] Der Kosinus des Winkels zwischen zwei Vektoren ergibt sich aus dem Quotienten vom Skalarprodukt der beiden Vektoren und dem Produkt der Betr\"age der Vektoren. \begin{align*}
    \cos{\varphi}&= \frac{\skalar{\vect{a}}{\vect{b}}}{\abs{\vect{a}}\abs{\vect{b}}} &\abs{\vect{a}}&\neq 0, \abs{\vect{b}}\neq 0
    \end{align*}
	  \end{rem}
	  \begin{rem}[Richtungskosinus] Ein Vektor $\vect{a}$ bildet mit den drei Koordinatenachsen der Reihe nach die Winkel $\alpha, \beta, \gamma$, die als \textit{Richtungswinkel} bezeichnet werden. Der Kosinus der jeweiligen Winkel wird als Richtungskosinus bezeichnet. \begin{align*}
	  \cos{\alpha}&=\frac{ \skalar{\vect{a}}{\vect{e}_1}}{\abs{\vect{a}}\abs{\vect{e}_1}}=\frac{a_x}{a} &\cos{\beta}&=\frac{ \skalar{\vect{a}}{\vect{e}_2}}{\abs{\vect{a}}\abs{\vect{e}_2}}=\frac{a_y}{a} &\cos{\gamma}&=\frac{ \skalar{\vect{a}}{\vect{e}_3}}{\abs{\vect{a}}\abs{\vect{e}_3}}=\frac{a_z}{a}
	  \end{align*}
	  Die Richtungswinkel sind jedoch nicht voneinander unabh\"angig, sondern \"uber die Beziehung \begin{align*}
	  \cos{\alpha}^2 + \cos{\beta}^2 + \cos{\gamma}^2 = 1
	  \end{align*}
	  miteinander verkn\"upft.
	  \end{rem}
	
	\item das \textbf{Vektorprodukt} (auch Kreuzprodukt) $\vect{a}\times \vect{b}$ hat als Ergebnis einen Vektor, der senkrecht auf $\vect{a}$ und $\vect{b}$ steht und dessen L\"ange gleich dem Produkt der Betr\"age von $\vect{a}, \vect{b}$ und dem Sinus des durch die Vektoren eingeschlossenen Winkels $\varphi$ ist. \begin{align*}
	\vect{a}\times \vect{b}&= \begin{pmatrix}
	y z' - z y' \\ z x' - x z' \\ x y' - y x' \end{pmatrix}
	\end{align*}
	Es gelten die Rechenregeln:
	  \begin{itemize}
	  \item Distributivgesetz: \begin{align*}
	  \vect{a}\times \left( \vect{b} + \vect{c}\right) &= \vect{a}\times \vect{b} + \vect{a} \times \vect{c} \\
	  \left( \vect{a}+  \vect{b}\right) \times \vect{c} &= \vect{a}\times \vect{c} + \vect{b} \times \vect{c}
	  \end{align*}
	  \item Anti-Kommutativgesetz: \begin{align*}
	  \vect{a}\times  \vect{b}&= - \left(\vect{b}\times  \vect{a} \right) 
	  \end{align*}
	  \item weitere Regeln: \begin{align*}
	  \lambda \left( \vect{a}\times \vect{b} \right) &= \left( \lambda \vect{a}\right) \times \vect{b} = \vect{a}\times \left( \lambda \vect{b}\right)
	  \end{align*}
	  \end{itemize}
\end{itemize}  

  \subsection{Gruppen \cite[S. 13]{Bosch2014}}
  Unter einer \textit{inneren Verkn\"upfung} auf einer Menge $\set{M}$ versteht man eine Abbildung $f: \set{M} \times \set{M} \to \set{M}$. Sie ordnet jedem Paar $(a, b)$ von Elementen aus
$\set{M}$ ein Element $f(a, b) \in \set{M}$ zu. Es wird die Notation $ a \cdot b$ anstelle von $f\of{a,b}$ verwendet, um den verkn\"upfenden Charakter der Abbildung zu verdeutlichen. 
  \begin{defn} Eine Menge $\set{G}$ mit einer inneren Verkn\"upfung $\set{G} \times \set{G} \to \set{G}, (a, b) \to a \cdot b$, hei\ss{}t eine Gruppe, wenn die folgenden Eigenschaften erf\"ullt
sind:
\begin{itemize}
\item Die Verkn\"upfung ist assoziativ, d. h. es gilt \begin{align}
(a \cdot b) \cdot c &= a \cdot (b \cdot c) \quad \forall { } a, b, c \in \set{G}. \label{gl:gruppenDef1}
\end{align}
\item Es existiert ein neutrales Element $e$ in $\set{G}$, das hei\ss{}t ein Element $e \in \set{G}$ mit \begin{align}
e \cdot a = a \cdot e = a \quad \forall { } a \in \set{G}. \label{gl:gruppenDef2}
\end{align}
\item Zu jedem $a \in \set{G}$ gibt es ein inverses Element, das hei\ss{}t ein Element $b \in \set{G}$ mit \begin{align}
a \cdot  b = b \cdot  a = e. \label{gl:gruppenDef3}
\end{align} Dabei ist $e$ das nach \eqnref{gl:gruppenDef2} existierende (eindeutig bestimmte) neutrale Element von $\set{G}$.
\item Die Gruppe hei\ss{}t kommutativ oder abelsch, falls die Verkn\"upfung kommutativ
ist, das hei\ss{}t falls zus\"atzlich gilt: \begin{align}
a \cdot b = b \cdot a \quad \forall { } a, b \in \set{G}. \label{gl:gruppenDef4}
\end{align}
\end{itemize}
  \end{defn}
  
\section{Starrk\"orperbewegung}\label{sec:starrkoerperbewegung}
Die Bewegung eines Punktes $p$ im euklidischen Raum wird durch die Angabe seiner Position in Bezug zu einem inertialen Koordinatensystem $\KOS{I}$ zu jedem Zeitpunkt $\acs{t}$ eindeutig beschrieben. Die Position des Punktes $p$ sei durch das Tripel $\left( x, y, z \right) \in \R^{3}$ gegeben. Die Trajektorie von $p$ kann dann durch die parametrisierte Bahn $p\of{t}=\left(x\of{t}, y\of{t}, z\of{t}\right) \in \R^{3} $ beschrieben werden. Da nicht die Bewegung von einzelnen Punkten, sondern die Bewegung eines Starrk\"orpers beschrieben werden soll, soll zun\"achst der Begriff Starrk\"orper definiert werden.

\begin{defn}[Starrk\"orper] Ein Starrk\"orper ist dadurch gekennzeichnet, dass die Distanz zweier beliebiger Punkte $p, q$, welche auf dem K\"orper liegen, unabh\"angig von der Bewegung des K\"orpers, immer konstant bleibt. Die anf\"angliche Position des Punktes $p$ sei beschrieben durch $p\of{0}$. Die Position nach einer beliebigen Zeit $\acs{t}$ (und einer beliebigen Bewegung) sei beschrieben durch $p\of{t}$. Die Nomenklatur gelte f\"ur den Punkt $q$ analog. F\"ur einen Starrk\"orper wird gefordert: \begin{align*}
\norm{p\of{t} - q\of{t}}&=\norm{p\of{0}-q\of{0}} = \text{konstant}
\end{align*}
\end{defn}
Eine Starrk\"orperbewegung kann prinzipiell aus Rotation, Translation oder einer \"Uberlagerung dieser Bewegungen bestehen. Wird ein K\"orper durch eine Teilmenge $O \in \R^{3}$ beschrieben, so kann seine Bewegung als eine kontinuierliche Zuordnung $g\of{t}: O \to R^{3}$ beschrieben werden. Die kontinuierliche Zuordnungsvorschrift $g\of{t}$ beschreibt, wie sich die einzelnen Punkte des K\"orpers relativ zu einem inertialen, festen Koordinatensystem mit Voranschreiten der Zeit $\acs{t}$ bewegen. Die Zuordnungsvorschrift $g$ darf dabei die Distanz zwischen Punkten des K\"orpers und die Orientierung von Vektoren, welche Punkte des K\"orpers verbinden, nicht ver\"andern. Damit ergibt sich die Definition einer Abbildung von Starrk\"orpern: 

\begin{defn}[Abbildung eines Starrk\"orpers] \cite{RichardM.Murray1994} Eine Zuordnungsvorschrift $g: \R^{3} \to \R^{3}$ ist die Abbildung eines Starrk\"orpers genau denn, wenn sie folgende Eigenschaften besitzt: \begin{enumerate}
\item Distanzen bleiben unver\"andert: $\norm{g\of{p}- g\of{q}}=\norm{p-q} \text{ f\"ur alle Punkte } p,q \in R^{3}$
\item Das Kreuzprodukt bleibt erhalten: $g\of{\vect{v}\times \vect{w}} = g\of{\vect{v}}\times g\of{\vect{w}} \text{ f\"ur alle Vektoren } \vect{v},\vect{w}\in R^{3}$.
\end{enumerate}
\end{defn}

\begin{rem} Man kann mit Hilfe der Polarisationsformel zeigen, dass das Skalarprodukt durch die Abbildungsvorschrift $g$ f\"ur einen Starrk\"orper erhalten bleibt \cite{RichardM.Murray1994}: \begin{align*}
\skalar{\vect{v}}{\vect{w}}&= \skalar{g\of{\vect{v}}}{g\of{\vect{w}}}.
\end{align*}
Ein orthonormales Rechtssystem wird durch die Abbildungsvorschrift $g$ demnach wieder in ein orthonormales Rechtssystem transformiert.
\end{rem}
Der Astronom und Mathematiker Giulio Mozzi zeigte bereits 1763, dass eine r\"aumliche Bewegung in eine Drehung und eine Verschiebung entlang der Drehachse zerlegt werden kann. Da sich die Teilchen eines Starrk\"orpers nicht relativ zueinander bewegen k\"onnen, kann die Bewegung eines Starrk\"orpers durch die relative Bewegung eines k\"orperfesten Koordinatensystems $\KOS{K}$ zu einem Inertialsystem beschrieben werden. Das Koordinatensystem $\KOS{K}$ erf\"ulle dabei die in \ref{ssec:SdT_mathGrundl_kos} genannten Eigenschaften. Das k\"orperfeste Koordinatensystem hat seinen Ursprung in einem beliebigen Punkt $p$ des K\"orpers. Die Orientierung von $\KOS{K}$ beschreibt die Rotation des K\"orpers und die Lage des Ursprungs von $\KOS{K}$ relativ zum Inertialsystem beschreibt den translatorischen Anteil der Starrk\"orperbewegung. Hat $\KOS{K}$ die Einheitsvektoren $\vect{v}_1, \vect{v}_2, \vect{v}_3$, dann kann die Bewegung von $\KOS{K}$ durch die Abbildung $g$ beschrieben werden. Genauer gesagt liefert $g\of{\vect{v}_1}, g\of{\vect{v}_2}, g\of{\vect{v}_3}$ die Orientierung von $\KOS{K}$ und $g\of{p}$ die Lage des Ursprungs nach einer Starrk\"orperbewegung. 
  \subsection{Rotationsmatrizen}
  Gegeben sei ein Koordinatensystem $\KOS{K}$, welches um eine Achse $w$ relativ zu einem inertialen Koordinatensystem $\KOS{I}$ gedreht wurde. Die Achsen von $\KOS{K}$ relativ zu $\KOS{I}$ seien gegeben durch die Vektoren $\vect{u}, \vect{w}, \vect{v} \in \R^{3}$. Die drei Spaltenvektoren werden horizontal zu einer Matrix \begin{align}
  \matr{R}&=\begin{bmatrix}
  \vect{u} & \vect{w} & \vect{v}
  \end{bmatrix} \label{gl:rotMatrDef}
  \end{align}
  zusammengefasst. Die Matrix $\matr{R}$ wird als Rotationsmatrix bezeichnet. 
    \subsubsection{Eigenschaften von Rotationsmatrizen}
    Die Spalten der Rotationsmatrix $\matr{R}\in \R^{3\times 3}$ seien die Vektoren $\vect{u}, \vect{w}, \vect{v} \in \R^{3}$. Da diese ein Koordinatensystem aufspannen haben sie die in \eqnref{gl:kosEvSenkr} und \eqnref{gl:kosEvRechtssys} definierten Eigenschaften. Aus \eqnref{gl:kosEvSenkr} folgt f\"ur die Matrix $\matr{R}$ \begin{align}
    \matr{R}\transp{\matr{R}} &= \transp{\matr{R}}\matr{R} = \matr{I} \label{gl:rotMatrTransp} 
    \intertext{und mit Hilfe der Regeln der linearen Algebra \cite[S. 100]{Papula2014} und \eqnref{gl:kosEvSenkr}}
    \det{\matr{R}} &= \skalar{\vect{u}}{\vect{w}\times \vect{v}}= \skalar{\vect{u}}{\vect{u}} =  1 \label{gl:rotMatrDet}
\end{align} Die Menge der orthogonalen $3 \times 3$ Matrizen mit der Determinante eins wird als $\set{SO}\of{3}$ bezeichnet \cite{RichardM.Murray1994}. Allgemein wird definiert: \begin{align}
\set{SO}\of{n} = \left \lbrace \matr{R}\in \R^{n\times n}: \matr{R}\transp{\matr{R}}=\matr{I}, \det{\matr{R}}=+1 \right \rbrace. \label{gl:mengeSoDef}
\end{align}
  \subsection{Koordinatensysteme}
	


