\chapter{Mathematische Grundlagen}\label{ch:mathGrundl}
\section{Einf\"uhrung}\label{sec:mathGrundl_einfuehrung}
    Bei der Analyse von Problemen ist es in der Mathematik  ebenso wie in anderen Fachbereichen \"ublich, mit Hilfe von Modellen m\"oglichst einfache Grundstrukturen zu finden, welche f\"ur die L\"osung des untersuchten Problems von Interesse sind. Dabei kann die gezielte Untersuchung eines einzelnen Modells losgel\"ost von der eigentlichen Problemstellung durchgef\"uhrt werden. Dadurch ist ein Modell in der Regel leichter \"uberschaubar, als das eigentliche Problem. \newline
    Als einf\"uhrendes Beispiel wird ein Modell des dreidimensionalen Raumes entworfen. Dieses ist in weiten Teilen \cite{Bosch2014} nachempfunden. \newline
    Die Beschreibung des dreidimensionalen Raumes baut auf einer Reihe von Grundstrukturen auf. Die Basis bilden Mengen, deren Elemente und Abbildungen (siehe Abschnitt \ref{sec:mathGrundl_mengen}). Darauf aufbauend werden im Abschnitt \ref{sec:mathGrundl_gruppen} Gruppen als Mengen mit einer inneren Abbildung  definiert. Im Abschnitt \ref{sec:mathGrundl_koerper} werden die Eigenschaften von Gruppen weiter eingeschr\"ankt, was auf den Begriff des K\"orper f\"uhrt. Anschlie\ss{}end k\"onnen Vektorr\"aume \"uber einem K\"orper und deren Elemente - die Vektoren - definiert werden (siehe \ref{sec:mathGrundl_vektorraeume}). Wegen ihrer hohen Relevanz f\"ur diese Arbeit werden im Abschnitt \ref{sec:mathGrundl_punkteVektoren} die Rechenregeln von Vektoren und Matrizen eingef\"uhrt. Au\ss{}erdem wird der Unterschied zwischen Punkten und Vektoren aufgef\"uhrt. Die Ausf\"uhrungen zu diesen Elementen der linearen Algebra sind \cite{Bosch2014}, \cite{MatthiasPlaue2009} und \cite{Modler2011d} entnommen. Die Eigenschaften von Vektoren sind \cite{Papula2014} entnommen. \newline
    Der dreidimensionale Raum \footnote{der dreidimensionale Raum wird in dieser Arbeit als Anschauungsraum bezeichnet}  wird auf Basis der so eingef\"uhrten Begriffe mit Hilfe  einer geeigneten Basis als Vektorraum \"uber dem K\"orper der reellen Zahlen in Form eines Koordinatensystems im Kapitel \ref{ch:kos} eingef\"urht.   \newline
    Um die Position eines K\"orpers im Raum beschreiben zu k\"onnen werden fernerhin f\"ur Vektoren und Matrizen unter anderem die Rechenregeln Addition und Multiplikation nach \cite{Papula2014} definiert. Im Kapitel \ref{ch:kos} werden die so eingef\"uhrten Gesetze verwendet um die Handhabung von Koordinatensystemen im Detail zu beschreiben. \newline
    
    
    Zur Beschreibung des dreidimensionalen Raumes sei eine Ebene E gegeben, welche beliebig im Anschauungsraum liegt. Weiterhin sei ein beliebiger Punkt dieser Ebene gegeben, welcher als Nullpunkt $O$ eines Koordinatensystems dienen soll. Zus\"atzlich seien Drei Geraden $X, Y, \text{und }Z$ gegeben, welche als Koordinatenachsen dienen. Alle Drei Geraden sollen sich dabei im Punkt $O$ schneiden und die reellen Zahlen komplett durchlaufen. Au\ss{}erdem sollen keine zwei Geraden parallel zueinander sein. Ferner sollen die Geraden paarweise derart senkrecht aufeinander stehen, dass sie ein Rechtssystem bilden. \"Uberdies sei auf jeder Gerade ein spezieller Punkt definiert: $I_{x}, I_{y}$ und $I_{z}$. Dieser Punkt habe vom Ursprung, entlang der jeweiligen Achse auf der er liegt, genau den Abstand 1. Man bezeichnet diesen Abstand als \textit{Einheitsl\"ange}. \newline
    Hinzukommen wird ein Vektor $\vect{e}_{x}$ definiert, welcher vom Ursprung aus auf den Punkt $I_{x}$ zeigt und damit zwangsl\"aufig auf der Geraden X liegt. Analog werden die Vektoren  $\vect{e}_{y}$ und $\vect{e}_{z}$ definiert. Diese Vektoren mit \textit{Einheitsl\"ange}, welche entlang der Koordinatenachsen liegen, werden \textit{Einheitsvektoren} genannt. Als Tripel notiert haben sie entlang der Achse, in welche  sie zeigen, eine 1 als Eintrag und sonst eine 0. Damit gilt $\vect{e}_{x}=\left( 1,0,0\right), \vect{e}_{y}=\left(0,1,0\right), \vect{e}_{z}=\left( 0,0,1\right)$. Durch die genannten Bedingungen ist es nicht m\"oglich einen der \textit{Einheitsvektoren} als \textit{Linearkombination} der Anderen darzustellen. Damit bilden diese \textit{Einheitsvektoren} die \textit{Basis} f\"ur einen \textit{Vektorraum} $V$. Da die \textit{Einheitsvektoren} jeweils Drei Komponenten besitzen ist der \textit{Vektorraum} gleich dem dreidimensionalen Raum. \newline
    Weiterhin sei ein Streckungsfaktor $\alpha \in \R$ definiert. Mit Hilfe von $\alpha \cdot I_{x}$ sei das Bild des Punktes $I_{x}$ beschrieben, welches sich durch Streckung mit Streckungszentrum im Ursprung $O$, entlang der x-Achse, um den Streckungsfaktor $\alpha$ ergibt. Die Zuordnung $\alpha \to \alpha \cdot I_{x}$ liefert damit eine eindeutige, umkehrbare Zuordnung der reellen Zahlen auf die Punkte der Geraden X. Dabei ist das Bild der Streckung des \textit{Einheitsvektors} $\vect{e}_{x}$ \"aquivalent mit dem Bild der Streckung des Punktes $I_{x}$. Zus\"atlich seien f\"ur die Achsen Y und Z die Streckungsfaktoren $\beta$ und $\gamma$ nach dem gleichen Schema definiert. \newline
        
    Mit Hilfe dieser Festlegungen k\"onnen beliebige Punkte im Raum beschrieben werden. Alle Punkte des Raumes bilden dabei die \textit{Menge} $R^{3}$. Man sagt, dass die Punkte \textit{Elemente} dieser \textit{Menge} sind. Betrachtet man zum Beispiel einen Punkt auf der Ebene E, so kann man dieses \textit{Element} als Tripel von reellen Zahlen interpretieren: $P=\left( x_{1}, y_{1}, z_{1}\right)$. Man nennt dieses Tripel die \textit{Koordinaten} von $P$ (bez\"uglich des gew\"ahlten Koordinatensystems). Der Ursprung des Koordinatensystems hat bez\"uglich des Koordinatensystems, dessen Ursprung er ist, immer die Koordinaten $(0,0,0)$. Den Wert von $x_{1}$ erh\"alt man geometrisch durch Konstruktionen einer Normalen bez\"uglich der x-Achse, welche den Punkt $P$ durchl\"auft. Den Fu\ss{}punkt dieser Normalen kann man durch Streckung des zuvor definierten Punktes $I_{x}$ um den Faktor $\alpha_{P}$ erhalten. Dabei entspricht eben diese reelle Zahl $\alpha_{P}$ dem Wert von $x_{1}$. \newline
    Die Werte f\"ur $y_{1}$ und $z_{1}$ ergeben sich analog. Die so erhaltene Zuordnung $P \to \left( x_{1}, y_{1}, z_{1}\right)$ bezeichnet man als eine Abbildung. Mit Hilfe dieser Abbildung wird beliebigen Punkten der Ebene E in eindeutiger Weise ein Zahlentripel von reellen Zahlen zugeordnet. Man sagt auch, dass dieses Zahlentripel ein \textit{Element} des \textit{Vektorraumes} $V$ ist, welcher \"uber dem \textit{K\"orper} der reellen Zahlen definiert ist. Da die Achsen des Koordinatensystems durch die \textit{Basis} des \textit{Vektorraumes} beschrieben werden sind die Zahlenwerte des Tripels zwangsl\"aufig abh\"angig von der Wahl der Koordinatensystemursprungs und der \textit{Basisvektoren}. \newline
    Da dieses Zahlentripel nicht nur ein \textit{Element} einer \textit{Menge}, sondern insbesondere ein Vektor eines \textit{Vektorraumes} ist, gibt es alternative Schreibweisen f\"ur den Punkt $P$. Die \"ublichste Alternative ist die Darstellung als Spaltenvektor $\vect{v} \in \R^{3}$. Dabei erh\"alt man $\vect{v}$ durch Subtraktion der Koordinaten des Punktes $P$ von den Koordinaten des gew\"ahlten Koordinatenursprungs. Beziehen sich alle Angaben auf das gleiche System, so entsprechen die Komponenten von $\vect{v}$ genau den Koordinaten von $P$ und die Darstellung als Spaltenvektor lautet: $\vect{v}=\begin{pmatrix} x_{1} \\ y_{1} \\ z_{1} \end{pmatrix}$. Man kann $\vect{v}$ auch als \textit{Linearkombination} der \textit{Basis} des \textit{Vektorraumes} $V$ darstellen: \begin{align*}
    \vect{v}&= x_{1} \vect{e}_{x} + y_{1} \vect{e}_{y} + z_{1} \vect{e}_{z} = \alpha_{P} \vect{e}_{x} + \beta_{P} \vect{e}_{y} + \gamma_{P} \vect{e}_{z}
\end{align*}      

  \begin{rem} Die Beschr\"ankung auf die Betrachtung von Rechtssystemen ist als einf\"uhrendes Beispiel besonders gut geeignet, da alle in dieser Arbeit verwendeten Koordinatensysteme die damit verbundenen Eigenschaften erf\"ullen sollen. Wollte man auch nicht orthogonale Koordinatensystem betrachten, so f\"uhrt dies zwangsl\"aufig auf die Betrachtung von kovarianten und kontravarianten Basen und die Tensorrechnung. Eine Einf\"uhrung in dieses Thema ist \cite{Roethlisberger2007} zu entnehmen. Eine ausf\"uhrliche Behandlung der Thematik ist in \cite{Jaenich2005} enthalten. 
  \end{rem}
  
   
  \section{Mengen}\label{sec:mathGrundl_mengen}
\begin{defn}[Mengen; nach G. Cantor] Eine Menge ist eine Zusammenfassung von bestimmten, wohlunterschiedenen Objekten unserer Anschauung oder unseres Denkens zu einem Ganzen. Die Objekte hei\ss{}en Elemente der Menge \cite{Cantor1895}. \newline
Ist $\set{M}$ eine Menge und $x$ ein Objekt, so notiert man $x \in \set{M}$, wenn die Menge $\set{M}$ das Objekt $x$ enth\"alt und $x \notin \set{M}$, wenn dies nicht der Fall ist. \newline
Enth\"alt die Menge $\set{M}$ keine Elemente, so nennt man dies die \textbf{leere Menge}\footnote{jede Menge besitzt die leere Menge als Teilmenge} $\lbrace  \rbrace$ beziehungsweise $\emptyset$.
\end{defn}
Mengen k\"onnen durch Aufz\"ahlung aller Elemente oder durch die Angabe von Eigenschaften, welche die Elemente erf\"ullen sollen, definiert werden. Ein Beispiel f\"ur verschiedene Mengendefinitionen ist im Beispiel \ref{ex:mathGrundl_mengen} zu finden. \newline  
Die angegebene Definition f\"ur Mengen ist zwar anschaulich, verzichtet aber auf eine axiomatische Begr\"undung. Im Rahmen dieser Arbeit ist diese Definition jedoch ausreichend. Ein pr\"azise Definition von Mengen erfordert erheblichen Aufwand und ist beispielsweise in \cite{Asser1975} enthalten. \newline
  Die folgenden Mengen sind von besonderer Bedeutung: \begin{itemize}
  \item die nat\"urlichen Zahlen $\set{N}=\lbrace 0, 1, 2, \dots \rbrace$
  \item die ganzen Zahlen $\set{Z}=\lbrace 0, \pm 1, \pm 2, \dots \rbrace$
  \item die rationalen Zahlen $\set{Q}=\lbrace \left. \frac{p}{q} \right| q,p \in \set{Z}, q \neq 0 \rbrace$
  \item die reellen Zahlen $\set{R}$ als Menge aller, unter Umst\"anden nicht abbrechenden, Dezimalbr\"uche \cite[S. 12]{MatthiasPlaue2009}
  \end{itemize}

\begin{exmp}[Mengendefinitionen]\label{ex:mathGrundl_mengen} \begin{align*}
\set{M}_{1}&= \lbrace 1,2,5,8,10 \rbrace \\
\set{M}_{2}&= \lbrace x | x \text{ ist eine ganze Zahl und ungerade} \rbrace
\end{align*}
\end{exmp}  
  
  Die Reihenfolge der Elemente einer Menge ist ohne Bedeutung. Daher gilt $\set{M}_{1}= \lbrace 1,4,8 \rbrace = \lbrace 4,8,1 \rbrace $. Enth\"alt eine Menge ein Element mehrfach, so ist diese Multiplizit\"at ohne Bedeutung, daher gilt $\set{M}_{1}= \lbrace 1,3,5 \rbrace = \lbrace 1,3,5,3,5 \rbrace$. Zur Handhabung von Mengen gibt es eine Reihe von Axiomen, auf welche im Folgenden eingegangen wird. 

\subsection{Teilmengen} Es sei $\set{X}$ eine Menge und $P\of{x}$ eine Aussage. Die G\"ultigkeit der Aussage (wahr oder falsch) sei f\"ur alle Elemente $x$ \"uberpr\"ufbar. Man nennt dann \begin{align}
\set{Y}&= \lbrace \left. x\in \set{X}\right| P\of{x} \text{ ist wahr} \rbrace \label{gl:mathGrundl_mengen_teilmenge}
\end{align} eine \textbf{Teilmenge} von $\set{X}$ und notiert $\set{Y} \subset \set{X}$. \newline
 Die Mengen $\set{X}$ und $\set{Y}$ hei\ss{}en \textit{gleich}, wenn jedes Element von $\set{X}$ auch in $\set{Y}$ enthalten ist und umgekehrt: $\set{X}=\set{Y} \Leftrightarrow \forall x \left( x \in \set{X} \Leftrightarrow x \in \set{Y} \right)$. \newline
  Gilt $\set{Y} \subset \set{X} \wedge \set{Y} \neq \set{X}$, so nennt man $\set{Y}$ eine \textit{echte Teilmenge} von $\set{X}$ und notiert $\set{Y} \subsetneq \set{X}$. \newline
  Die Gesamtheit aller Teilmengen einer Menge $\set{X}$ bildet die sogenannte \textbf{Potenzmenge} $\set{P}\of{\set{X}} = \lbrace \set{U}| \set{U} \subset \set{X} \rbrace $.  

\begin{rem}[Notation f\"ur Teilmengen] In der Literatur ist die Kennzeichnung einer echten Teilmenge im Gegensatz zu einer Teilmenge nicht einheitlich. Es gibt folgende Notationen: \begin{itemize}
\item Teilmenge: $\subset$, echte Teilmenge: $\subsetneq$
\item Teilmenge: $\subseteq$, echte Teilmenge: $\subset$
\end{itemize}
\end{rem}

  \subsection{Vereinigung und Durchschnitt}
  Sei $\set{X}$ eine Menge und $\set{I}$ eine Indexmenge, das hei\ss{}t die Elemente von $\set{I}$ sollen als Indizes dienen. Ist f\"ur jedes $i \in \set{I}$ eine Teilmenge $\set{X}_{i} \subset \set{X}$ gegeben, so nennt man  \begin{align}
  \underset{{i\in\set{I}}}{\bigcup} \set{X}_{i} &:= \lbrace \left. x \in \set{X} \right| \text{ es existiert ein } i \in \set{I} \text{ mit } x\in \set{X}_{i}\rbrace \label{gl:mathGrundl_mengen_vereinigung}
  \end{align}
  die \textbf{Vereinigung} der Mengen $\set{X}_{i}, i \in \set{I}$. In der so entstehenden Teilmenge von $\set{X}$ sind also all jene $x$ enthalten, die in wenigstens einer der vereinten  Teilmengen $\set{X}_{i}$ enthalten sind. \newline
  Der \textbf{Durchschnitt}\footnote{den \textit{Durchschnitt} von Mengen bezeichnet man auch kurz als \textit{Schnitt}} der Mengen wird mit 
  \begin{align}
  \underset{{i\in\set{I}}}{\bigcap} \set{X}_{i} &:= \lbrace \left. x \in \set{X} \right| x\in \set{X}_{i} \quad \forall  i \in \set{I} \rbrace \label{gl:mathGrundl_mengen_durchschnitt}
  \end{align}
  beschrieben. In der durch den Durchschnitt gebildeten Teilmenge von $\set{X}$ sind also diejenigen Elemente $x$ der Menge $\set{X}$ enthalten, welche in allen Teilmengen $\set{X}_{i}$ enthalten sind, von denen der Durchschnitt gebildet wurde. \newline
  F\"ur die Verkn\"upfung von zwei Mengen schreibt man insbesondere: \begin{align*}
  \set{X}_{1} \cup \set{X}_{2} &= \lbrace \left. x \right| x \in \set{X}_{1} \lor x \in \set{X}_{2} \rbrace
  \end{align*}
  f\"ur die Vereinigung und \begin{align*}
  \set{X}_{1} \cap \set{X}_{2} &= \lbrace \left. x \right| x \in \set{X}_{1} \land x \in \set{X}_{2} \rbrace
  \end{align*}
  f\"ur den Durchschnitt. \newline
  Ist der Durchschnitt zweier Mengen leer, gilt also \begin{align}
  \set{X}_{1} \cap \set{X}_{2} = \emptyset \label{gl:mathGrundl_mengen_disjunkt}
\end{align} dann bezeichnet man die Mengen als \textbf{disjunkt}.

  \subsection{Differenz von Mengen} Sind $\set{X}_{1}$ und $\set{X}_{2}$ Teilmengen einer Menge $\set{X}$, so hei\ss{}t \begin{align}
  \set{X}_{1} - \set{X}_{2} &:= \lbrace \left.x \in \set{X}_{1}\right| x \notin \set{X}_{2} \rbrace \label{gl:mathGrundl_mengen_differenz}
  \end{align}
  die \textbf{Differenz} von $\set{X}_{1}$ und $\set{X}_{2}$. Auch dies ist eine Teilmenge von $\set{X}$, sogar von $\set{X}_{1}$.
  
  
  \subsection{Kartesisches Produkt}
   Es seien $\set{X}_{1}, \dots, \set{X}_{n}$ Mengen. Dann hei\ss{}t \begin{align}
  \underset{i=1}{\overset{n}{\prod}}&= \lbrace \left.\left( x_{1}, \dots, x_{n} \right)\right| x_{1} \in \set{X}_{1}, \dots, x_{n} \in \set{X}_{n}  \rbrace \label{gl:mathGrundl_mengen_kartProd}
  \end{align}
  das \textbf{kartesische Produkt} der Mengen $\set{X}_{1}, \dots, \set{X}_{n}$. Eine gleichbedeutenden Notation lautet $\set{X}_{1} \times \dots \times \set{X}_{n}$. Die Elemente $\left( x_{1}, \dots, x_{n} \right)$ werden als \textbf{n-Tupel} mit Komponenten $x_{i} \in \set{X}_{i}, i=1, \dots, n$, bezeichnet. \newline
  Zwei n-Tupel $\left( x_{1}, \dots, x_{n} \right), \left( x'_{1}, \dots, x'_{n} \right)$ gelten genau dann als gleich, wenn $x_{i}=x'_{i}$ f\"ur $i=1, \dots, n$ erf\"ullt ist. \newline
  Das 2-Tupel $(3,5)$ beschreibt beispielsweise einen Punkt auf einer Zahlenebene. \newline  
  Ein 3-Tupel bezeichnet man als \textit{Tripel}. 

  \subsection{M\"achtigkeit}
   Sei $\set{X}$ eine Menge, dann bezeichnet man mit \begin{align}
  \abs{\set{X}} \label{gl:mathGrundl_mengen_maechtigkeit}
  \end{align} die \textbf{M\"achtigkeit} der Menge und meint damit die Anzahl der Elemente, welche in $\set{X}$ enthalten sind. Es gilt \begin{align*}
  \abs{\set{X}} &:= \begin{cases}n, \text{ falls $\set{X}$ endlich ist und $n$ Elemente enth\"alt} \\ \infty, \text{ falls $\set{X}$ nicht endlich ist.} \end{cases}
\end{align*}   

  \subsection{Rechengesetze f\"ur Mengen} Kommutativit\"at usw. 
  siehe \cite[S.14]{MatthiasPlaue2009}
  
  \section{Abbildung}\label{sec:mathGrundl_abbildung}
  \begin{defn}[Abbildung] Eine Abbildung $f: \set{X} \longrightarrow \set{Y}$ zwischen zwei Mengen $\set{X}$ und $\set{Y}$ ist eine Vorschrift, welche jedem $x \in \set{X}$ ein wohlbestimmtes Element $y \in \set{Y}$ zuordnet, welches mit $f\of{x}$ bezeichnet mit. Man schreibt auch $x \longmapsto f\of{x}$. Man bezeichnet $\set{X}$ als den Definitionsbereich und $\set{Y}$ als den Bild- oder Wertebereich\footnote{der Bildbereich wird auch Zielmenge genannt} der Abbildung $f$.
  \end{defn}
  \subsection{Komposition von Abbildungen}
  Gegeben seien zwei Abbildung $f: \set{X} \longrightarrow \set{Y}$ und $g: \set{Y} \longrightarrow \set{Z}$ zwischen Mengen. Dann kann man die Abbildungen komponieren: \begin{align}
  g \circ f &: \set{X} \longrightarrow \set{Z}, &x&\longmapsto g\of{f\of{x}}. \label{gl:mathGrundl_abbildung_komposition}
  \end{align}
  Alternative Bezeichnungen f\"ur eine Komposition lauten \textit{Hintereinanderausf\"uhrung} und \textit{Verkettung}.
  
  \begin{exmp}[Ort als Bild der Zeit] Ordnet man jedem Zeitpunkt $t$ den Ort beziehungsweise die Position des Schwerpunkts eines K\"orpers zu, so entspricht dies einer Abbildung $f: t \longmapsto f\of{t}$ Der Definitionsbereich sei beispielsweise gegeben durch beliebige, positive, reelle Zahlen und der Wertebereich durch beliebige, reelle Zahlen. Der K\"orper kann sich dann frei im Anschauungsraum bewegen und seine Position ist mit Hilfe der Abbildung $f\of{t}$ eindeutig beschrieben. 
  \end{exmp}
  
  \begin{rem}[Umkehrbarkeit einer Abbildung] In der Definition einer Abbildung wird gefordert, dass jedem $x \in \set{X}$ ein eindeutiges Bild $y \in \set{Y}$ zuordnet wird. Die Umkehrung, dass jedem $y \in \set{Y}$ ein eindeutiges Urbild $x \in \set{X}$ zugeordnet werden kann, kann daraus nicht geschlussfolgert werden.
  \end{rem}
  
  Seien $\set{M}\subset \set{X}, \set{N}\subset \set{Y}$ und $f: \set{X}\longrightarrow\set{Y}$ eine Abbildung, so nennt man: \begin{align}
  f\of{\set{M}}&:= \lbrace \left.y\in\set{Y}\right| \text{ es existiert ein } x\in\set{M} \text{ mit } y=f\of{x} \rbrace \label{gl:mathGrundl_abbildung_bild}
  \end{align} das \textbf{Bild}(-menge) von $\set{M}$ unter der Abbildung $f$ und \begin{align}
  f^{-1}\of{\set{N}}&:= \lbrace \left.x\in\set{X}\right| f\of{x}\in \set{N} \rbrace \label{gl:mathGrundl_abbildung_urbild}
  \end{align} das \textbf{Urbild}(-menge) von $\set{N}$ unter der Abbildung $f$. \newline
  Weiterhin bezeichnet man $f$ als 
  \begin{itemize}
  \item \textbf{injektiv}, falls f\"ur $x_{1},x_{2}\in \set{X}$ gilt: $f\of{x_{1}}=f\of{x_{2}}\implies x_{1}=x_{2}$, das hei\ss{}t, dass verschiedene Elemente des Definitionsbereichs auf verschiedene Elemente des Wertebereichs abgebildet werden beziehungsweise dass das Urbild $f^{-1}\of{y}$ eines jeden $y\in\set{Y}$ entweder leer ist, oder aus genau einem $x\in\set{X}$ besteht
  \item \textbf{surjektiv}, falls gilt: $\forall y \in \set{Y} \quad \exists x \in\set{X}: f\of{x}=y$  , das hei\ss{}t jedes Element des Wertebereichs wird durch Abbildung mindestens eines Elements aus dem Definitionsbereich erreicht
  \item \textbf{bijektiv}, falls $f$ injektiv und surjektiv ist. F\"ur bijektive Abbildungen $f$ l\"asst sich die \textbf{Umkehrabbildung} $g: \set{Y}\longrightarrow \set{X}, g:=f^{-1}$ bilden, welche jedem Element der Bildmenge ein Element aus dem Definitionsbereich zuordnet. 
  \end{itemize}
  
  \subsection{Eigenschaften von Abbildungen}
  siehe \cite{Modler2011d}, insbesondere f\"ur \textit{Kompositionen} \cite[S. 37]{MatthiasPlaue2009}
    
  
   \section{Gruppen}\label{sec:mathGrundl_gruppen}
  Unter einer \textbf{inneren Verkn\"upfung} auf einer Menge $\set{M}$ versteht man eine Abbildung $f: \set{M} \times \set{M} \to \set{M}$. Sie ordnet jedem Paar $(a, b)$ von Elementen aus $\set{M}$ ein Element $f(a, b) \in \set{M}$ zu. Die innere Verkn\"upfung muss also eine abgeschlossene Verkn\"upfung sein. Es wird die Notation $ a \cdot b$ anstelle von $f\of{a,b}$ verwendet, um den verkn\"upfenden Charakter der Abbildung zu verdeutlichen. \newline
  Ist die Verkn\"upfung kommutativ, gilt also $f\of{a,b}=f\of{b,a}$ f\"ur alle $a,b \in \set{M}$, so wird die Verkn\"upfung $f$ durch $a+b$ notiert. 
  
  \begin{defn} Eine Menge $\set{G}$ mit einer inneren Verkn\"upfung $\circ: \set{G} \times \set{G} \longrightarrow \set{G},$ \hfill \newline  $(a, b) \longmapsto a \circ b$, hei\ss{}t eine Gruppe $\of{\set{G}, \circ}$, wenn die folgenden Eigenschaften erf\"ullt
sind:
\begin{itemize}
\item Die Verkn\"upfung $\circ$ ist assoziativ, d. h. es gilt \begin{align}
(a \circ b) \circ c &= a \circ (b \circ c) \quad \forall { } a, b, c \in \set{G}. \label{gl:mathGrundl_gruppen_assoziativ}
\end{align}
\item Es existiert ein \textbf{neutrales Element} $e$ in $\set{G}$, das hei\ss{}t ein Element $e \in \set{G}$ mit \begin{align}
e \circ a = a \circ e = a \quad \forall { } a \in \set{G}. \label{gl:mathGrundl_gruppen_neutrElem}
\end{align} Das neutrale Element einer Gruppe ist offensichtlich zugleich links-neutral und rechts-neutral.
\item Zu jedem $a \in \set{G}$ gibt es ein \textbf{inverses Element}, das hei\ss{}t ein Element $b \in \set{G}$ mit \begin{align}
a \circ  b = b \circ  a = e. \label{gl:mathGrundl_gruppen_invElemt}
\end{align} Dabei ist $e$ das nach \eqnref{gl:mathGrundl_gruppen_neutrElem} existierende (eindeutig bestimmte) neutrale Element von $\set{G}$. Das inverse Element $b$ einer Gruppe ist links-invers und recht-invers.
\item Die Gruppe $\of{\set{G}, \circ}$ hei\ss{}t kommutativ oder \textbf{abelsch}, falls die Verkn\"upfung kommutativ ist, das hei\ss{}t falls zus\"atzlich gilt: \begin{align}
a \circ b = b \circ a \quad \forall { } a, b \in \set{G}. \label{gl:mathGrundl_gruppen_kommutativ}
\end{align}
\end{itemize}
  \end{defn}  
  \begin{exmp}[Beispiele f\"ur Gruppen]\hfill \newline
  \begin{itemize}
  \item Die Menge $\set{Z}$ mit der Verkn\"upfung Addition \glqq $+$\grqq { }
  \item Die Menge $\set{R}$ mit der Verkn\"upfung Addition \glqq $+$\grqq { } und $\set{R}^{\ast}:=\set{R}-\lbrace 0 \rbrace $ mit der Multiplikation \glqq$\cdot$\grqq { }
  \end{itemize}
  
\end{exmp}    
  
  \begin{rem}[multiplikative Verkn\"upfungen] Ist die Verkn\"upfung einer Gruppe in multiplikativer Schreibweise gegeben, so wird \begin{itemize}
  \item das neutrale Element $e$ als \textbf{Einselement} bezeichnet und als 1 notiert
  \item das inverse Element $b$ zu $a$ als $a^{-1}$ notiert
  \item das Verkn\"upfungszeichen \glqq$\cdot$\grqq { }meist weggelassen
  \item f\"ur endliche Elemente $a_{1}, \dots, a_{n}\in\set{G}$ das Produkt der Elemente als \hfill \newline $\underset{i=1}{\overset{n}{\prod}}a_{i}:=a_{1} \cdot \dots \cdot a_{n}$ definiert, wobei $\underset{i=1}{\overset{0}{\prod}}a_{i}:= 1$ gilt.
  \end{itemize}
\end{rem}    

\begin{rem}[additive Verkn\"upfungen] Ist die Verkn\"upfung einer Gruppe kommutativ, so verwendet man die additive Schreibweise und \begin{itemize}
  \item bezeichnet das neutrale Element $e$ als \textbf{Nullelement} und schreibt 0
  \item notiert das inverse Element $b$ zu $a$ als $-a$
  \item definiert f\"ur endliche Elemente $a_{1}, \dots, a_{n}\in\set{G}$ die Summe der Elemente als \hfill \newline $\underset{i=1}{\overset{n}{\sum}}a_{i}:=a_{1} + \dots + a_{n}$, wobei $\underset{i=1}{\overset{0}{\sum}}a_{i}:= 0$ gilt.
  \end{itemize}
\end{rem} 
  
  \section{K\"orper}\label{sec:mathGrundl_koerper}
  K\"orper sind Zahlsysteme mit gewissen Axiomen f\"ur die Addition und Multiplikation, welche auf den Axiomen von Gruppen aufbauen. \newline
  \begin{defn}[K\"orper]\label{def:mathGrundl_koerper} Ein K\"orper ist eine Menge $\set{K}$ mit zwei inneren Verkn\"upfungen, geschrieben als Addition und Multiplikation, so dass folgende Bedingungen erf\"ullt sind: \begin{enumerate}
  \item $\set{K}$ ist eine abelsche Gruppe bez\"uglich der Addition, das hei\ss{}t die Addition ist assoziativ nach \eqnref{gl:mathGrundl_gruppen_assoziativ}, hat ein neutrales Element 0 nach \eqnref{gl:mathGrundl_gruppen_neutrElem} und ein inverse Element $-a\in \set{K}$ nach \eqnref{gl:mathGrundl_gruppen_invElemt} zu $a\in\set{K}$ und sie ist kommutativ nach \eqnref{gl:mathGrundl_gruppen_kommutativ}
  \item $\set{K}^{\ast}=K\backslash \lbrace 0 \rbrace$ ist eine abelsche Gruppe bez\"uglich der Multiplikation, das hei\ss{}t die Multiplikation ist assoziativ nach \eqnref{gl:mathGrundl_gruppen_assoziativ}, hat das neutrale Element 1 nach \eqnref{gl:mathGrundl_gruppen_neutrElem} und das inverse Element $a^{-1}$ nach \eqnref{gl:mathGrundl_gruppen_invElemt} f\"ur $a, a^{-1} \in \set{K}$ und sie ist kommutativ nach \eqnref{gl:mathGrundl_gruppen_kommutativ}
  \item Es gelten die Distributivgesetze: \begin{align*}
  a \cdot \left( b + c \right) &= a \cdot b + a \cdot c \\
  \left( a + b \right) \cdot c &= a\cdot c + b \cdot c 
  \end{align*} f\"ur $a,b,c \in \set{K}$
  \end{enumerate}
  Wird ein K\"orper mit den zugeh\"origen Verkn\"upfungen angegeben, so kann man $\set{K}$ als Tripel $\left(\set{K}, +, \cdot \right) $ notieren.
  \end{defn}
  \begin{rem}[Abgeschlossenheit] Entsprechend der Definition \ref{def:mathGrundl_koerper} f\"ur einen K\"orper $\set{K}$ bilden die Rechenoperationen Addition und Multiplikation Elemente des K\"orpers auf Elemente des K\"orpers ab. Da $\set{K}$ nie verlassen wird spricht man in diesem Zusammenhang von der \textbf{Abgeschlossenheit} des K\"orpers $\set{K}$.
\end{rem} 
\begin{rem}[K\"orper und Gruppen] Ist $\set{K}$ ein K\"orper mit den Verkn\"upfungen Addition und Multiplikation, dann sind $\left(\set{K}, + \right)$ und $\left(\set{K} - \lbrace 0 \rbrace, \cdot \right)$ Gruppen. \newline
Die Menge aller invertierbaren $n \times n$ Matrizen \"uber einem K\"orper mit Matrizenmultiplikation bildet die \textit{allgemeine lineare Gruppe}.
\end{rem}
   
 \begin{exmp}[K\"orper] H\"aufig verwendete K\"orper sind \begin{itemize}
 \item die Menge der rationalen Zahlen $\set{Q}$ mit den Verkn\"upfungen Addition und Multiplikation
 \item  die Menge der reellen Zahlen $\set{R}$ mit den Verkn\"upfungen Addition und Multiplikation
 \item die Menge $\set{R}\times \set{R}:= \lbrace \left.(x, y)\right| x,y \in \set{R} \rbrace$ der Paare (2-Tupel) reeller Zahlen mit den Verkn\"upfungen Addition und Multiplikation. Mit den Verkn\"upfungen \begin{align*}
 \left( x_{1}, y_{1}\right)+ \left( x_{2}, y_{2}\right)&= \left( x_{1}+x_{2}, y_{1}+y_{2}\right)\\
 \left( x_{1}, y_{1}\right) \cdot \left( x_{2}, y_{2}\right)&= \left( x_{1}x_{2}-y_{1}y_{2}, x_{1}y_{2}+x_{2}y_{1}\right)\\
 \end{align*} und den neutralen Elementen der Addition: $(0,0)$ und Multiplikation: $(1,0)$ bildet diese Menge einen K\"orper. Jedes Zahlenpaar $\left(x, y \right)$ kann als Linearkombination dargestellt werden: \begin{align*}
 \left(x, y \right) &= x \cdot \left( 1,0 \right) + y \cdot \left( 0, 1 \right).
\end{align*} F\"uhrt man folgende Schreibweisen ein: \begin{itemize}
\item $\left( 1,0 \right): 1$
\item $\left( 0,1 \right): i$ und bezeichnet $i$ als imagin\"are Einheit,
\end{itemize} so erh\"alt man die Darstellung $\left(x, y \right) \equiv x + i y$. Die so erhaltenen, reellen Zahlenpaare bezeichnet man als die \textbf{komplexen Zahlen}, welche in \textit{algebraischer Normalform} vorliegen. Der K\"orper der komplexen Zahlen lautet damit $\left(\set{C},+,\cdot \right)$.
 \end{itemize}
 Die Menge der nat\"urlichen Zahlen $\set{N}$ bildet mit der Addition \textit{keinen} K\"orper, denn es gibt beispielsweise kein inverses Element bez\"uglich der Addition f\"ur das Element $5$. F\"ur das zu erwartende, inverse Element $-5$ gilt offensichtlich $-5\notin \set{N}$. \newline
 Die Menge der ganzen Zahlen $\set{Z}$ bildet mit der Addition zwar eine abelsche Gruppe aber keinen K\"orper, denn f\"ur das Element $8$ gibt es kein inverses Element $b$ bez\"uglich der Multiplikation, f\"ur welches $b\in\set{Z}$ gilt.   
 \end{exmp}
  
  \subsection{Weitere Eigenschaften und Begriffe}
  insbesondere Homomorphismen f\"ur Gruppen und K\"orper \cite[S. 86, 87, 89, 93]{Modler2011d} \hfill \newline
  Kern und Bild eines Homomorphismus \cite[S. 86, 93!]{Modler2011d}
  
  \section{Vektorr\"aume}\label{sec:mathGrundl_vektorraeume}
  Ein Vektorraum ist durch eine abelsche Gruppe mit der inneren Verkn\"upfung Addition $\left(\set{V}, +\right)$ (Abschnitt \ref{sec:mathGrundl_gruppen}), einen K\"orper $\set{K}$ (Abschnitt \ref{sec:mathGrundl_koerper}) mit Skalaren als Elementen, eine \"au\ss{}ere Multiplikation, welche Elemente von $\set{K}$ mit denen von $\set{V}$ verkn\"upft und auf $\set{V}$ abbildet, und vier Vertr\"aglichkeitsgesetzen - den so genannten Vektorraumaxiomen -  gegeben. Der K\"orper $\set{K}$ entspricht h\"aufig den reellen Zahlen $\set{R}$ oder den komplexen Zahlen $\set{C}$. 
  \begin{defn}[$\set{K}-$Vektorraum] Es sei $\set{K}$ ein K\"orper, $\left(\set{V}, +\right)$ eine abelsche Gruppe und \begin{align}
  \cdot : & \set{K}\times \set{V} \longrightarrow V, &\left( \lambda, \vect{x}\right) &\longmapsto \lambda \cdot \vect{x}
  \end{align}
  eine Abbildung. Ferner seien die Vektoren $\vect{x},\vect{y},\vect{z}\in \set{V}$ und die Skalare $\lambda, \mu \in \set{K}$ gegeben. Man nennt $\set{V}$ einen Vektorraum \"uber dem K\"orper $\set{K}$ oder kurz einen $\set{K}-$Vektorraum genau dann, wenn die folgenden Eigenschaften erf\"ullt sind: \begin{align}
  \intertext{Assoziativit\"at der Multiplikation mit Skalaren} 
  \left( \lambda \mu \right)\cdot \vect{x}&= \lambda \cdot \left( \mu \cdot \vect{x} \right)\label{gl:mathGrundl_vektorraeume_assoz}\\
  \intertext{Distributivit\"at}   
  \lambda \cdot \left( \vect{x}+\vect{y} \right)&=\lambda \cdot \vect{x} + \lambda \cdot \vect{y} \label{gl:mathGrundl_vektorraeume_distr1} \\
  \intertext{und} 
  \left( \lambda + \mu \right) \cdot \vect{x} &= \lambda \cdot \vect{x}+ \mu \cdot \vect{x} \label{gl:mathGrundl_vektorraeume_distr2}  \\
  \intertext{F\"ur die Multiplikation gibt es ein neutrales 1-Element}
  1\cdot \vect{x}&= \vect{x} \label{gl:mathGrundl_vektorraeume_neutrElem}  
  \end{align}
  \end{defn}
  Die Vektorraumaxiome nach \eqnref{gl:mathGrundl_vektorraeume_assoz}, \eqnref{gl:mathGrundl_vektorraeume_distr1}, \eqnref{gl:mathGrundl_vektorraeume_distr2} und \eqnref{gl:mathGrundl_vektorraeume_neutrElem} beschreiben eine Vertr\"aglichkeit der zwei Verkn\"upfungen Addition und Multiplikation des $\set{K}-$Vektorraums. \hfill \newline 
    Die Elemente eines Vektorraumes werden als \textbf{Vektoren} bezeichnet. Vektoren werden allein durch ihre Eigenschaften definiert. Typische Beispiele f\"ur Vektoren sind Elemente der  Vektorr\"aume $\set{R}^{2}$ und $\set{R}^{3}$, aber auch Funktionen k\"onnen Vektoren sein. Ein Vektor ist also nicht zwangsl\"aufig ein Pfeil mit L\"ange und Richtung. \hfill \newline
  Das neutrale Element der Addition der abelschen Gruppe $\set{V}$ wird als Nullvektor bezeichnet und mit $\vect{0}$ gekennzeichnet. Ist aus dem Kontext erkennbar, dass es sich um einen Nullvektor handelt, so wird die besondere Notation als Vektor h\"aufig weggelassen und man schreibt $0$ f\"ur den Nullvektor. 
 \begin{rem}[unendliche Vektorr\"aume] F\"ur jeden K\"orper $\set{K}$ und jede nat\"urliche Zahl $n$ ist die Menge \begin{align*}
 \set{V}&=\set{K}^{n} = \left\lbrace \left.\begin{pmatrix}
 v_{1} \\ \vdots \\ v_{n}
\end{pmatrix} \right| v_{1}, \dots, v_{n} \in \set{K}  \right\rbrace
 \end{align*} mit komponentenweiser Addition und Multiplikation mit Skalaren aus $\set{K}$ ein $\set{K}-$Vektorraum. Insbesondere werden somit f\"ur den K\"orper der reellen Zahlen $\set{R}$ der reelle Vektorraum $\set{R}^{n}$ und f\"ur den K\"orper der komplexen Zahlen $\set{C}$ der komplexe Vektorraum $\set{C}^{n}$ definiert. 
 \end{rem}
  
  
  
  \section{Punkte und Vektoren}\label{sec:mathGrundl_punkteVektoren}
  Zur Beschreibung eines Punktes wird ein Koordinatensystem ben\"otigt. Ein Punkt wird eindeutig durch seine \textit{Position} relativ zu diesem Koordinatensystem beschrieben. Als Koordinatensystem wird das Koordinatensystem $\KOS{I}$ aus \ref{sec:kos_rechtssys} verwendet. Die Position eines Punktes $p$ kann dann wie folgt beschrieben werden: \begin{align*}
  p &=  \left(x | y | z\right)\in \R^{3} = x  \vect{e}_1 + y \vect{e}_2 + z \vect{e}_3 
  \end{align*} Die Position von $p$ ist damit relativ zu $\KOS{I}$ eindeutig beschrieben. \newline
    \subsection{Vektoren}
  Unter einem Vektor $\vect{a}\in \R^{3}$ versteht man einen Pfeil, welcher durch seine \textit{Richtung} und seinen \textit{Betrag} beziehungsweise seine L\"ange eindeutig charakterisiert wird. Die Begriffe Betrag und L\"ange werden im Folgenden synonym verwendet. Mit beiden Eigenschaften ist die euklidische Norm des Vektors $\vect{a}$ gemeint. Die folgende Darstellung wird f\"ur den Betrag eines Vektors $\vect{a}$ verwendet:
  \begin{align*}
  \abs{\vect{a}} \equiv a := \norm{\vect{a}}
  \intertext{F\"ur einen Vektor $\vect{a}$ gilt immer:}
  \abs{\vect{a}} &\geq 0
  \end{align*}
  Beschreibt ein Vektor die Auspr\"agung einer physikalischen Gr\"o\ss{}e, so geh\"ort zu deren vollst\"andiger Beschreibung zus\"atzlich zu Betrag und Richtung noch die Angabe einer Ma\ss{}einheit. Ein typisches Beispiel ist der Betrag einer Kraft $\vect{F}: \abs{\vect{F}}= 100 \cdot N$ \newline
Bei der Arbeit mit Vektoren wird zwischen verschiedenen Typen unterschieden \cite[S. 26]{Riessinger2007j}:
\begin{itemize}
\item \textit{Freie Vektoren}\footnote{\textit{freie Vektoren} werden auch als \textit{Richtungsvektoren} bezeichnet} k\"onnen beliebig im Raum verschoben werden, so lange Richtung und Betrag konstant bleiben. Damit sind parallele Verschiebungen und Verschiebungen entlang der Wirkungslinie m\"oglich. Dieser Typ von Vektoren wird in der Regel gemeint, wenn man von Vektoren spricht. Ein typisches Beispiel sind die Einheitsvektoren $\vect{e}$, mit denen die Koordinatenachsen eines Koordinatensystems beschrieben werden.
\item \textit{Gebundene Vektoren} beziehen sich auf einen festen Ursprung, von dem aus sie abgetragen werden. Ein typisches Beispiel daf\"ur ist der Ortsvektor $\vect{r}$ eines Raumpunktes $R$, welcher von einem spezifischen Koordinatenursprung $O$ aus angetragen wird: $\vect{r}=P-O = \vect{OP}$. In diesem Zusammenhang bezeichnet man den Punkt $R$ h\"aufig als \glqq Punkt R mit dem Ortsvektor $\vect{r}$\grqq { }und notiert wahlweise $P \equiv \vect{r}$.
\end{itemize}
Da die Notation von Punkten im Sinne eines Ortsvektors und die Notation von Richtungsvektoren identisch ist, kommt es leicht zu Verwechslungen. Erschwerend kommt hinzu, dass die Konzepte von Punkt und Vektor in der Literatur h\"aufig nicht gesondert betrachtet werden. Es ist damit die Aufgabe des Lesers sich klar zu machen, ob mit $\vect{a}$ ein Richtungsvektor oder ein Punkt gemeint ist. Verwendet man zur Darstellung Homogene Koordinaten (siehe \ref{sec:kos_homKoord}) so ist die Unterscheidung von Punkten und Vektoren offensichtlich. \newline
F\"ur die Darstellung von Vektoren gibt es verschiedene M\"oglichkeiten. Eine Variante ist die bereits verwendete Angabe von Anfangspunkt $q$ und Endpunkt $p$
\begin{align*}
\vect{a}&=  p - q \equiv \vect{qp}.
\end{align*}
Bei \textit{freien Vektoren} ist die Wahl der Anfangs- und Endpunkte jedoch nicht eindeutig. Es gibt also Punkte $r, s$, f\"ur die gilt:
\begin{align*}
\vect{a}&= p - q = r - s 
\intertext{mit}
p&\neq r \text{ und } q\neq s
\end{align*}
F\"ur einen \textit{gebundenen Vektor} gibt es keine solche Punkte $r, s$. Durch die Angabe eines Bezugspunktes, die Richtung und den Betrag des Vektors ist auch der Endpunkt eindeutig bestimmt. \newline
Eine weitere Darstellung f\"ur Vektoren ist die Komponentenschreibweise bez\"uglich eines Koordinatensystems $\KOS{I}$, welche durch Projektion auf die Basisvektoren von $\KOS{I}$ gegeben ist. 
\begin{align*}
\vect{\tensor*[_I]{a}{}}&=  \vect{\tensor*[_I]{a}{_x}} + \vect{\tensor*[_I]{a}{_y}}+ \vect{\tensor*[_I]{a}{_z}} =  x \vect{\tensor*[_I]{e}{_1}} + y \vect{\tensor*[_I]{e}{_2}}+ z \vect{\tensor*[_I]{e}{_3}} = 
\begin{pmatrix} x \\ y \\ z 
\end{pmatrix} 
\intertext{damit ergibt sich der Betrag eines Vektor zu}
\abs{\vect{\tensor*[_I]{a}{}}}&= \sqrt{x^2 + y^2 + z^2} \\
&=\sqrt{\skalar{\vect{a}}{\vect{a}}}
\intertext{Wenn bekannt ist, dass mit $\vect{a}$ ein Vektor gemeint ist, so notiert man den Betrag des Vektors in Kurzform mit $a$.}
\end{align*}
Man bezeichnet die Skalare $x, y, z$ als die \textit{Komponenten}, \textit{Koordinaten} oder auch \textit{Eintr\"age} des Vektors $\vect{\tensor*[_I]{a}{}}$. Der linksseitige Index $I$ kennzeichnet das verwendete Bezugssystem. Ist das Bezugssystem eindeutig, so wird dieser Index weggelassen.\newline

Folgende Eigenschaften gelten f\"ur Vektoren $\vect{a}=\begin{pmatrix} x \\ y \\ z \end{pmatrix}, \vect{b}=\begin{pmatrix} x' \\ y' \\ z' \end{pmatrix}$ (nach \cite{Papula2014}):

\begin{itemize}
	\item Vektoren sind \textit{gleich}, wenn sie in Richtung und Betrag \"ubereinstimmen \begin{align*}
	\vect{a}=\vect{b} &\Leftrightarrow \abs{a}=\abs{b} \wedge \vect{a} \uparrow \uparrow \vect{b}
	\end{align*}
	
	\item \textbf{Addition}/ Subtraktion von Vektoren erfolgt durch Addition/ Subtaktion der Komponenten
	\begin{align*}
	\vect{a}\pm\vect{b}&=\begin{pmatrix} x \\ y \\ z \end{pmatrix} \pm \begin{pmatrix} x' \\ y' \\ z' \end{pmatrix} = \begin{pmatrix} x\pm x' \\ y\pm y' \\ z\pm z' \end{pmatrix} 
	\end{align*}
	Es gelten die Rechenregeln:
	  \begin{itemize}
	  \item Kommutativgesetz: \begin{align*}
	  \vect{a}\pm\vect{b} &= \pm\vect{b} + \vect{a}
	  \end{align*}
	  \item Assoziativgesetz: \begin{align*}
	  \vect{a}+ \left( \vect{b}+\vect{c} \right) &= \left( \vect{a}+\vect{b} \right) + \vect{c}
	  \end{align*}
	  \end{itemize}
	  
	\item Multiplikation mit einem Skalar $\lambda, \mu \in \R$ erfolgt durch Multiplikation aller Komponenten mit dem Skalar \begin{align*}
	\lambda\cdot \vect{a}&= \begin{pmatrix} \lambda\cdot x \\ \lambda\cdot y \\ \lambda\cdot z \end{pmatrix}
	\end{align*}
	Es gelten die Rechenregeln:
	  \begin{itemize}
	  \item Distributivgesetz: \begin{align*}
	  \lambda \left( \vect{a}+\vect{b}\right) &= \lambda\vect{a} + \lambda\vect{b}
	  \end{align*}
	  \item weitere Regeln: \begin{align*}
	  \left(\lambda + \mu \right) \vect{a}&= \lambda\vect{a} + \mu\vect{a} \\
	  \left(\lambda \mu \right) \vect{a}&= \lambda \left( \mu  \vect{a} \right) =  \mu \left(\lambda  \vect{a} \right)\\ 
	  \abs{\lambda \vect{a}}&= \abs{\lambda}\abs{\vect{a}}
	  \end{align*}
	  \end{itemize}
	
	\item Das \textbf{Skalarprodukt} \acs{skalarProd} zweier Vektoren ist das Produkt der Betr\"age und dem Kosinus des von den Vektoren eingeschlossenen Winkels $\varphi$ \begin{align*}
	\skalar{\vect{a}}{\vect{b}}&= \abs{a}\abs{b}\cos{\varphi} = \left( x \vect{e}_1 + y \vect{e}_2 + z \vect{e}_3\right) \left( x' \vect{e}_1 + y' \vect{e}_2 + z' \vect{e}_3 \right)
	\end{align*}
	Es gelten die Rechenregeln:
	  \begin{itemize}
	  \item Kommutativgesetz: \begin{align*}
	  \skalar{\vect{a}}{\vect{b}} &= \skalar{\vect{b}}{\vect{a}}
	  \end{align*}
	  \item Distributivgesetz: \begin{align*}
	  \skalar{\vect{a}}{\vect{b}+\vect{c}} &= \skalar{\vect{a}}{\vect{b}} + \skalar{\vect{a}}{\vect{c}}
	  \end{align*}
	  \item weitere Regeln: \begin{align*}
	  \lambda \skalar{\vect{a}}{\vect{b}} &= \skalar{\lambda \vect{a}}{\vect{b}} = \skalar{ \vect{a}}{\lambda\vect{b}}
	  \end{align*}
	  \end{itemize}
	  \begin{rem}[Orthogonale Vektoren] Verschwindet das Skalarprodukt zweier von Null verschiedenen Vektoren, so stehen diese senkrecht aufeinander. \begin{align*}
	  \skalar{\vect{a}}{\vect{b}}&=0 \Leftrightarrow \vect{a} \perp  \vect{b}
    \end{align*}	   
    \end{rem}
    \begin{rem}[Winkel zwischen Vektoren] Der Kosinus des Winkels zwischen zwei Vektoren ergibt sich aus dem Quotienten vom Skalarprodukt der beiden Vektoren und dem Produkt der Betr\"age der Vektoren. \begin{align*}
    \cos{\varphi}&= \frac{\skalar{\vect{a}}{\vect{b}}}{\abs{\vect{a}}\abs{\vect{b}}} &\abs{\vect{a}}&\neq 0, \abs{\vect{b}}\neq 0
    \end{align*}
	  \end{rem}
	  \begin{rem}[Richtungskosinus] Ein Vektor $\vect{a}$ bildet mit den drei Koordinatenachsen seines Bezugssystems der Reihe nach die Winkel $\alpha, \beta, \gamma$, die als \textit{Richtungswinkel} bezeichnet werden. Der Kosinus der jeweiligen Winkel wird als Richtungskosinus bezeichnet. \begin{align*}
	  \cos{\alpha}&=\frac{ \skalar{\vect{a}}{\vect{e}_1}}{\abs{\vect{a}}\abs{\vect{e}_1}}=\frac{a_x}{a} &\cos{\beta}&=\frac{ \skalar{\vect{a}}{\vect{e}_2}}{\abs{\vect{a}}\abs{\vect{e}_2}}=\frac{a_y}{a} \\
	  \cos{\gamma}&=\frac{ \skalar{\vect{a}}{\vect{e}_3}}{\abs{\vect{a}}\abs{\vect{e}_3}}=\frac{a_z}{a}
	  \end{align*}
	  Die Richtungswinkel sind jedoch nicht voneinander unabh\"angig, sondern \"uber die Beziehung \begin{align*}
	  \cos{\alpha}^2 + \cos{\beta}^2 + \cos{\gamma}^2 = 1
	  \end{align*}
	  miteinander verkn\"upft.
	  \end{rem}
	
	\item das \textbf{Vektorprodukt} (auch Kreuzprodukt) $\vect{a}\times \vect{b}$ hat als Ergebnis einen Vektor, der senkrecht auf $\vect{a}$ und $\vect{b}$ steht und dessen L\"ange gleich dem Produkt der Betr\"age von $\vect{a}, \vect{b}$ und dem Sinus des durch die Vektoren eingeschlossenen Winkels $\varphi$ ist. \begin{align*}
	\vect{a}\times \vect{b}&= \left( \abs{\vect{a}}\abs{\vect{b}} \sin\of{\vartheta}\right) \vect{n} =  \begin{pmatrix}
	y z' - z y' \\ z x' - x z' \\ x y' - y x' \end{pmatrix}
	\end{align*} Dabei ist $\vect{n}$ derjenige zu $\vect{a}$ und $\vect{b}$ senkrechte Einheitsvektor, der diese zu einem Rechtssystem erg\"anzt. \hfill \newline
	Es gelten die Rechenregeln:
	  \begin{itemize}
	  \item Distributivgesetz: \begin{align*}
	  \vect{a}\times \left( \vect{b} + \vect{c}\right) &= \vect{a}\times \vect{b} + \vect{a} \times \vect{c} \\
	  \left( \vect{a}+  \vect{b}\right) \times \vect{c} &= \vect{a}\times \vect{c} + \vect{b} \times \vect{c}
	  \end{align*}
	  \item Anti-Kommutativgesetz: \begin{align*}
	  \vect{a}\times  \vect{b}&= - \left(\vect{b}\times  \vect{a} \right) 
	  \end{align*}
	  \item weitere Regeln: \begin{align*}
	  \lambda \left( \vect{a}\times \vect{b} \right) &= \left( \lambda \vect{a}\right) \times \vect{b} = \vect{a}\times \left( \lambda \vect{b}\right)
	  \end{align*}
	  \end{itemize}
	  Da das Kreuzprodukt mit dem Vektor $\vect{a}$ eine lineare Abbildung ist, kann $\vect{b} \to \vect{a} \times \vect{b}$ mit Hilfe einer Matrix dargestellt werden: \begin{align}
	  \hat{\matr{a}} &= \begin{pmatrix}
	  0 & -z & y \\ z & 0 & -x \\ -y & x & 0
	  \end{pmatrix} \label{gl:SdT_mathGrundl_punkteVektoren_kreuzProdMatrix}\\
	  \vect{a}\times \vect{b} &= \hat{\matr{a}} \vect{b} = \begin{pmatrix}
	  0 & -z & y \\ z & 0 & -x \\ -y & x & 0
	  \end{pmatrix} \cdot \begin{pmatrix} x' \\ y' \\ z' \end{pmatrix} = \begin{pmatrix}
	y z' - z y' \\ z x' - x z' \\ x y' - y x' \end{pmatrix} \label{gl:SdT_mathGrundl_punkteVektoren_kreuzProdOp}
\end{align}	   
\end{itemize}  
