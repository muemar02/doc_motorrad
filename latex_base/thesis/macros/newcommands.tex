% % redefine \textmu to other mu commands usefull inside text
% \renewcommand{\textmu}{$\upmu$}

% automatische Eintraege von Literaturverwaltungssoftware braucht \mathplus als Befehl
\providecommand\mathplus{+}


% Defintionen, Theoreme
\newtheorem{defn}{Definition}
\newtheorem{satz}{Satz}
\newtheorem{hsatz}{Hilfssatz}

\newtheoremstyle{style1} 
   {}                   %Space above 
   {}                   %Space below 
   {}                      %Body font: original {\normalfont} 
   {}                      %Indent amount (empty = no indent, 
                           %\parindent = para indent) 
   {\normalfont\itshape}  %Thm head font original {\normalfont\bfseries} 
   {.}                     %Punctuation after thm head original : 
   { }              %Space after thm head: " " = normal interword 
                           %space; \newline = linebreak 
   {}%\textbf{\thmname{#1}\thmnumber{ #2}\thmnote{ (#3)}}}                     
                                        %Thm head spec (can be left empty, meaning 
                           %`normal') original {\underline{\thmname{#1}\thmnumber{ #2}\thmnote{ (#3)}}} 
		
\newtheoremstyle{style2} 
   {3pt}                   %Space above 
   {3pt}                   %Space below 
   {}                      %Body font: original {\normalfont} 
   {}                      %Indent amount (empty = no indent, 
                           %\parindent = para indent) 
   {\normalfont\bfseries}  %Thm head font original {\normalfont\bfseries} 
   {}                     %Punctuation after thm head original : 
   {\newline}              %Space after thm head: " " = normal interword 
                           %space; \newline = linebreak 
   {\textbf{\thmname{#1}\thmnumber{ #2}\thmnote{ (#3)}}}                     
                                        %Thm head spec (can be left empty, meaning 
                           %`normal') original {\underline{\thmname{#1}\thmnumber{ #2}\thmnote{ (#3)}}} 

\declaretheorem[name={Beispiel},style=style1,qed=$\lozenge$,numberwithin=chapter]{exmp}
\declaretheorem[name={Gegenbeispiel},style=style1,qed=$\lozenge$,numberwithin=chapter]{gegenexmp}
\declaretheorem[name={Übungsaufgabe},style=style2,numberwithin=chapter]{uea}
\declaretheorem[name={Bemerkung},style=style1,qed=$\blacklozenge$,numberwithin=chapter]{rem}

% Vektor
\DeclareRobustCommand{\vect}[1]{\overrightarrow{#1}}   %Vektor
\DeclareRobustCommand{\vectHomog}[1]{{\overrightarrow{#1}}^{H}}   %Vektor in homog Darstellung
\DeclareRobustCommand{\matr}[1]{\bm{#1}} %Matrix
%\newcommand{\vect}[1]{\boldsymbol #1}
\newcommand{\skalar}[2]{\left \langle #1,#2 \right \rangle} %Skalarprodukt
\newcommand{\transp}[1]{#1^{\text{T}}}
\DeclareRobustCommand{\inv}[1]{\ensuremath{#1^{-1}}}

% Caption with defined width
\newcommand{\wcaption}[2]{%
   \begin{minipage}{#1}%
   \caption{#2}%
   \end{minipage}%
}

\newcommand*{\rechterWinkel}[3]{% #1 = point, #2 = start angle, #3 = radius
   \draw[shift={(#2:#3)}] (#1) arc[start angle=#2, delta angle=90, radius = #3];
   \fill[shift={(#2+45:#3/2)}] (#1) circle[radius=1.25\pgflinewidth];
}

\newcommand{\figureref}[1]{(Abbildung \ref{#1})}%
\newcommand{\eqnref}[1]{Gleichung (\ref{#1})}%

% Command for margin text with usefull style
%\newcommand{\marginlabel}[1]{\mbox{}\marginline{\hspace{0pt}\footnotesize\sffamily #1}}%
\newcommand{\marginlabel}[1]{\marginnote{#1}}%

%\newcommand{\comment}[1]{\marginnote{#1}}%

% Enable space for figures that extent into the margin (right and/or leftside)
% Can be used inside a figure
% Note: sidecap defines a similar environment 'wide' !
\newenvironment{widespace}[2]{%
   \begin{list}{}{%
      \setlength{\topsep}{0pt}%
      \setlength{\leftmargin}{#1}%
      \setlength{\rightmargin}{#2}%
      \setlength{\listparindent}{\parindent}%
      \setlength{\itemindent}{\parskip}%
   }%
   \item[]%
}%
{%
   \end{list}%
}%

\newlength{\marginwidth}
\setlength{\marginwidth}{\marginparwidth}
\addtolength{\marginwidth}{\marginparsep}

%% Beispiel:
% \begin{figure}
% \begin{widespace}{-\marginwidth}{0pt}
%  \subfloat[Bergzebrastute]
%  {\includegraphics[width=0.45\linewidth]{../Bilder/Eingewoehnung2.jpg}}
%  \hspace*{1em}
%  \subfloat[Morro Moco]
%  {\includegraphics[width=0.45\linewidth]{../Bilder/bergzebra2.jpg}}
% \end{widespace}
% \end{figure}


% quantum optics - Latex Commands: Math **********************************
% ------------------------------------------------------------------------
% by: Matthias Pospiech
%%%%%%%%%%%%%%%%%%%%%%%%%%%%%%%%%%%%%%%%%%%%%%%%%%%%%%%%%%%%%%%%%%%%%%%%%%


% --| Math |-------------------------------------------------------

% -- Replacements --
\newcommand{\comp}{\ast}
%\renewcommand{\dagger}{+}

% -- Footnode --
\renewcommand{\thefootnote}{\arabic{footnote}}

% -- new commands --
\providecommand{\abs}[1]{\lvert#1\rvert}
\providecommand{\Abs}[1]{\left\lvert#1\right\rvert}
%\providecommand{\norm}[1]{\left\Vert#1\right\Vert}
\DeclareRobustCommand{\norm}[1]{\left\Vert#1\right\Vert}
\providecommand{\Trace}[1]{\ensuremath{\Tr\{\,#1\,\}}} % Trace /Spur
\providecommand{\KOS}[1]{\ensuremath{\mathit{#1}}} %Darstellung fuer Koordinatensysteme
\providecommand{\of}[1]{\ensuremath{\left(#1 \right)}} %parametrischer Zusammenhang einer Groesse
%

% -- differentials --
\renewcommand{\d}{\partial\mspace{2mu}} % partial diff
\newcommand{\td}{\,\mathrm{d}}	% total diff
\newcommand{\ddt}[1]{\frac{\td #1}{\td t}}

% -- Abbrevitations --
\renewcommand{\Re}{\text{Re}}			% Real value
\renewcommand{\Im}{\text{Im}}			% Real value
\newcommand{\complex}{\mathbb{C}} % Complex
\newcommand{\real}{\mathbb{R}}    % Real
%\newcommand{\R}{\real}						% Real
%\newcommand{\N}{\mathbb{N}}
%\newcommand{\Z}{\mathbb{Z}}
\DeclareRobustCommand{\set}[1]{\mathcal{#1}} %Darstellung von Mengen
\renewcommand{\L}{\set{L}}
\newcommand{\N}{\set{N}}
\newcommand{\R}{\set{R}}
\newcommand{\D}{\set{D}}

%
\newcommand{\Ham}{\mathcal{H}}    % Hamilton
\newcommand{\Prob}{\mathscr{P}}    
\newcommand{\unity}{\mathds{1}}   % 1-Vektor
\newcommand{\Fou}{\mathscr{F}}
%

\newcommand\gammab{\gamma_\bot}
\newcommand\gammap{\gamma_\parallel}
\newcommand\gammai{\gamma_\text{int}}
\newcommand\gammae{\gamma_\text{ext}}

% -- New Operators --
\DeclareMathOperator{\rot}{rot}
\DeclareMathOperator{\grad}{grad}
%\DeclareMathOperator{\div}{div}
\renewcommand{\div}{\text{div}\,}
\DeclareMathOperator{\Tr}{Tr}
\DeclareMathOperator{\const}{const}
\DeclareMathOperator{\Rang}{Rang}
\DeclareMathOperator{\Spur}{Spur}
\DeclareMathOperator{\Inter}{int}
\DeclareMathOperator*{\argmin}{arg\,min}
\DeclareMathOperator{\bd}{bd}
\DeclareMathOperator{\cl}{cl}
\DeclareMathOperator{\diag}{diag}
\DeclareMathOperator{\galois}{GF}
\DeclareMathOperator{\Cov}{Cov}
\DeclareMathOperator{\Var}{Var}
\DeclareMathOperator{\E}{E}
\DeclareMathOperator{\Span}{span}
\DeclareMathOperator{\Image}{im}
\DeclareMathOperator{\erf}{erf}
\DeclareMathOperator{\erfc}{erfc}

\newcommand{\vlaplace}[1][]{\mbox{\setlength{\unitlength}{0.1em}%
                            \begin{picture}(10,20)%
                              \put(3,2){\circle{4}}%
                              \put(3,4){\line(0,1){12}}%
                              \put(3,18){\circle*{4}}%
                              \put(10,7){#1}
                            \end{picture}%
                           }%
                     }%

\newcommand{\vLaplace}[1][]{\mbox{\setlength{\unitlength}{0.1em}%
                            \begin{picture}(10,20)%
                              \put(3,2){\circle*{4}}%
                              \put(3,4){\line(0,1){12}}%
                              \put(3,18){\circle{4}}%
                              \put(10,7){#1}
                            \end{picture}%
                           }%
                     }%

% -- new symbols --
%\newcommand{\laplace}{\Delta}
\newcommand{\dalembert}{\Box}

% -- new arrows --
\renewcommand{\leadsto}{\Longrightarrow}
\newcommand{\leftrightleadsto}{\Longleftrightarrow}


% -- Text subscripts--
\newcommand{\rel}{_\text{rel}}
%\newcommand{\st}{\text{st}}
%

% -- other --
\newcommand{\com}[2]{\underbrace{#1}_{\textrm{\scriptsize #2}}}
\newcommand{\with}[1]{\stackrel{\ref{#1}}{\Longrightarrow}}
%\newcommand{\unit}[1]{\,\textrm{#1}}

%\newcommand{\variance}[1]{\delta \mean{#1}^2}
\newcommand{\variance}[1]{(\Delta{#1})^2}
%\newcommand{\variance}[1]{\delta #1^2}

%Signum Funktion
\DeclareMathOperator\sgn{sgn}
% -- Physics --------------------------------
\newcommand\op[1]{{\hat{\mathrm{#1}}}}  % Operator

\newcommand\expect[1]{\ensuremath{\left\langle{#1}\right\rangle}} %
%
\newcommand{\mean}[1]{\ensuremath{\overline{#1}}} % mean value
%
\newcommand{\state}[1]{\ensuremath{\ket{#1}}}
%
\newcommand\commutator[2]{\ensuremath{\mathinner{%
    \mathopen[\,#1,#2\,\mathclose]}}}
\newcommand{\Commutator}[2]{\ensuremath{\left[\,#1,#2\,\right]}}
\newcommand{\bigcommutator}[2]{\ensuremath{\bigl[\,#1,#2\,\bigr]}}
\newcommand{\Bigcommutator}[2]{\ensuremath{\Bigl[\,#1,#2\,\Bigr]}}
%
\newcommand\poisson[2]{\mathinner{%
    \mathopen\{#1,#2\mathclose\}}}
%

% -- Layout --------------------------------

\newcommand*{\dashfill}{\leavevmode\cleaders\hbox{-}\hfill\kern0pt}

\newcommand*{\midhrulefill}{
\leavevmode
\cleaders\hbox to 1ex{\raisebox{.5ex}{\rule{1ex}{.4pt}}}\hfill\kern0pt
}

% -- Doppeltes Unterstreichen -------------
\newcommand\uli[2][]{
 \underline{#2\vphantom{#1}}
}

\newcommand\ulalign[2]{\uli[#2]{#1}&\uli[#1]{\;=#2}}
\newcommand\dulalign[2]{\underline{\uli[#2]{#1}}&\underline{\uli[#1]{\;=#2}}}


% -- SI-Einheiten -------------------------
\newcommand{\acrounit}[1]{
\acroextra{\makebox[18mm][l]{\si{#1}}}
}

% -- Leerseite
\newcommand\leerseite{\newpage\thispagestyle{empty}\hspace{1cm}\newpage}

% -- Anhangsinhaltsverzeichnis ------------
\makeatletter
% Die folgende Anweisung wird vermutlich irgendwann in scrlfile.sty eingebaut.
% Bis dahin ist es notwendig, sie selbst zu definieren, damit man innerhalb
% von \BeforeClosingMainAux \addtocontents verwenden kann:
\providecommand{\protected@immediate@write}[3]{%
  \begingroup
    \let\thepage\relax
    #2%
    \let\protect\@unexpandable@protect
    \edef\reserved@a{\immediate\write#1{#3}}%
    \reserved@a
  \endgroup
  \if@nobreak\ifvmode\nobreak\fi\fi
}
 
% Die folgende Umgebung wird verwendet, um innerhalb der toc-Datei einzelne
% Bereiche ein- und ausschalten zu können. In die toc-Datei wird die Umgebung
% dabei jeweils als \begin{tocconditional}{BEREICH}...\end{tocconditional}
% eingefügt.
\newenvironment*{tocconditional}[1]{%
  \expandafter\ifx\csname if@toccond@#1\expandafter\endcsname
                  \csname iftrue\endcsname
  \else
    \value{tocdepth}=-10000\relax
  \fi
  \typeout{tocdepth in `#1': \the\c@tocdepth}%
}{%
}
 
% Gleich nach dem Öffnen der toc-Datei beginnen wir den Haupt-Bereich "main":
\AtBeginDocument{%
  \addtocontents{toc}{\string\begin{tocconditional}{main}}
  \hypersetup{
    pdftitle = {\@title},
    pdfauthor = {\@author}
  }
}
% Und der letzte Bereich endet am Ende der toc-Datei.
\BeforeClosingMainAux{%
  \begingroup
    \let\protected@write\protected@immediate@write
    \addtocontents{toc}{\string\end{tocconditional}}%
  \endgroup
}
 
% Hier können nun neue Bereiche definiert ...
\newcommand*{\newtocconditional}[2][false]{%
  \expandafter\newif\csname if@toccond@#2\endcsname
  \csname @toccond@#2#1\endcsname
}
% Und ein- oder ausgeschaltet werden:
\newcommand*{\settocconditional}[2]{%
  \csname @toccond@#1#2\endcsname
}
 
% Neben dem Hauptbereich ...
\newtocconditional[true]{main}
% definieren wir noch einen Bereich für den Anhang.
\newtocconditional{appendix}
 
% Mit dem Anhang geben wir einerseits das Anhangsverzeichnis aus,
% andererseits beenden wir den aktuellen Bereich in der toc-Datei und beginnen
% den neuen Bereich "appendix". Damit im Haupt-Inhaltsverzeichnis ein Eintrag
% für das Anhangsverzeichnis erscheint, verwenden wir \addchap und zwar noch
% bevor der letzte Bereich geschlossen wird. Wenn wir es ganz sicher machen
% wollten, müssten wir die auskommentierten Zeilen noch aktivieren. So
% verlassen wir uns einfach darauf, dass vor dem appendix-Bereich der
% main-Bereich lag.
\g@addto@macro\appendix{%
%  \addtocontents{toc}{\string\end{tocconditional}^^J
%    \string\begin{tocconditional}{main}}%
  \addchap{Anhang}%
  \addtocontents{toc}{\string\end{tocconditional}
    \string\begin{tocconditional}{appendix}}%
  \appendixtableofcontents
}
 
% Jetzt definieren wir das Anhangsverzeichnis selbst als Alias für die
% toc-Datei. Dabei wird aber der Hauptbereich "main" deaktiviert und der
% Anhangsbereich "appendix" aktiviert.
\newcommand*{\appendixtableofcontents}{%
  \showtoc[{ %
    \aliastoc{\tocstyleTOC}{toc}%
    \settocconditional{main}{false}%
    \settocconditional{appendix}{true}%
  }]{toc}%
}
 
% Auch wenn man einen Anhang normalerweise nicht beenden kann, so ist es
% ggf. erwünscht, dass Literaturverzeichnis, Index etc. zwar nach den Kapiteln
% des Anhangs kommen, aber dem Hauptverzeichnis zugeordnet werden sollen. Also
% benötigen wir eine Anweisung, um in der toc-Datei den aktuellen Bereich zu
% beenden und wieder einen Hauptbereich einzuschalten:
\newcommand*{\postappendix}{%
  \addtocontents{toc}{\string\end{tocconditional}^^J%
      \string\begin{tocconditional}{main}}%
  \clearpage
}
 
\makeatother