%% Dokumentenklasse (Koma Script) -----------------------------------------
\documentclass[%
   %final,      % fertiges Dokument
	 % --- Paper Settings ---
   paper=a4,%
   paper=portrait, % landscape
   pagesize, % driver
   % --- Base Font Size ---
   fontsize=13bp,%
	 % --- Koma Script Version ---
   version=last, %
 ]{scrreprt} % Classes: scrartcl, scrreprt, scrbook

 %%%% Dokument/PDF Metadaten
\title{Studienarbeit}
\author{Marius M\"uller}

% Encoding der Dateien (sonst funktionieren Umlaute nicht)
% Fuer Linux -> utf8
% Fuer Windows, alte Linux Distributionen -> latin1

% Empfohlen latin1, da einige Pakete mit utf8 Zeichen nicht
% funktionieren, z.B: listings, soul.
%\usepackage[latin1]{inputenc}
%\usepackage[ansinew]{inputenc}
\usepackage[utf8]{inputenc}
%\usepackage{ucs}
%\usepackage[utf8x]{inputenc}

%%% Preambel
\input{latex_base/thesis/preambel/settings}


\input{latex_base/general/preambel/preambel-commands}

\usepackage{etex}
\usepackage[utf8]{inputenc}
\usepackage[T1]{fontenc}
\usepackage{lmodern}
\usepackage[english]{babel}
% hat nur mit .avi Datein funktioniert, nicht mit .mp4, extrem schlechtes Rendering
%\usepackage{multimedia} 
\usepackage{media9}
\usepackage{animate}
\usepackage{tabularx}
\usepackage{xcolor}
\usepackage{listings}
\usepackage{graphicx}
\usepackage{array}
\usepackage{colortbl}
\usepackage{fixltx2e}
\usepackage{textcomp}
\usepackage{eurosym}

\tolerance=1000
 
\usetheme[section,navigation,pagenum,ddc]{tud}
\useinnertheme[shadow=true]{rounded}
%\usetheme{Boadilla}
%\usecolortheme{tud}

% ~~~~~~~~~~~~~~~~~~~~~~~~~~~~~~~~~~~~~~~~~~~~~~~~~~~~~~~~~~~~~~~~~~~~~~~~
% Math Packages
% ~~~~~~~~~~~~~~~~~~~~~~~~~~~~~~~~~~~~~~~~~~~~~~~~~~~~~~~~~~~~~~~~~~~~~~~~
%%% Doc: ftp://tug.ctan.org/pub/tex-archive/macros/latex/required/amslatex/math/amsldoc.pdf
% Amsmath - Mathematik Basispaket
% fuer pst-pdf displaymath Modus vor pst-pdf benoetigt.
\usepackage{amsmath} %

\usepackage{units}

%%% Doc: ftp://tug.ctan.org/pub/tex-archive/macros/latex/contrib/mh/doc/mathtools.pdf
\usepackage[fixamsmath,disallowspaces]{mathtools}

%%% Doc: http://www.ctan.org/info?id=fixmath
\usepackage{fixmath}

%%% Doc: ftp://tug.ctan.org/pub/tex-archive/macros/latex/contrib/onlyamsmath/onlyamsmath.dvi
% Warnt bei Benutzung von Befehlen die mit amsmath inkompatibel sind.
\usepackage[
	all,
	warning
]{onlyamsmath}

\usepackage{bm}

\bibliographystyle{ieeetr}

\AtBeginDocument{\renewcommand*{\partname}{Theme}}

\setbeamerfont{description item}{series=\bfseries}
\setbeamertemplate{date/place in footline}[default][J. Wurm]
\setbeamertemplate{page number in footline}[frame][total]
\setlength{\tudbeamerfooterplacewidth}{0.3\linewidth}%
\setlength{\tudbeamerfooterpagenumwidth}{5em}%
\makeatletter
\setlength{\tudbeamerfootertitlewidth}{\paperwidth-\beamer@leftmargin-\beamer@rightmargin-\tudbeamerfooterplacewidth-\tudbeamerfooterpagenumwidth}%

\makeatother

\input{latex_base/general/preambel/Fonts}
%
%%%% Neue Befehle
% % redefine \textmu to other mu commands usefull inside text
% \renewcommand{\textmu}{$\upmu$}

% automatische Eintraege von Literaturverwaltungssoftware braucht \mathplus als Befehl
\providecommand\mathplus{+}


% Defintionen, Theoreme
\newtheorem{defn}{Definition}
\newtheorem{satz}{Satz}
\newtheorem{hsatz}{Hilfssatz}

\newtheoremstyle{style1} 
   {}                   %Space above 
   {}                   %Space below 
   {}                      %Body font: original {\normalfont} 
   {}                      %Indent amount (empty = no indent, 
                           %\parindent = para indent) 
   {\normalfont\itshape}  %Thm head font original {\normalfont\bfseries} 
   {.}                     %Punctuation after thm head original : 
   { }              %Space after thm head: " " = normal interword 
                           %space; \newline = linebreak 
   {}%\textbf{\thmname{#1}\thmnumber{ #2}\thmnote{ (#3)}}}                     
                                        %Thm head spec (can be left empty, meaning 
                           %`normal') original {\underline{\thmname{#1}\thmnumber{ #2}\thmnote{ (#3)}}} 
		
\newtheoremstyle{style2} 
   {3pt}                   %Space above 
   {3pt}                   %Space below 
   {}                      %Body font: original {\normalfont} 
   {}                      %Indent amount (empty = no indent, 
                           %\parindent = para indent) 
   {\normalfont\bfseries}  %Thm head font original {\normalfont\bfseries} 
   {}                     %Punctuation after thm head original : 
   {\newline}              %Space after thm head: " " = normal interword 
                           %space; \newline = linebreak 
   {\textbf{\thmname{#1}\thmnumber{ #2}\thmnote{ (#3)}}}                     
                                        %Thm head spec (can be left empty, meaning 
                           %`normal') original {\underline{\thmname{#1}\thmnumber{ #2}\thmnote{ (#3)}}} 

\declaretheorem[name={Beispiel},style=style1,qed=$\lozenge$,numberwithin=chapter]{exmp}
\declaretheorem[name={Gegenbeispiel},style=style1,qed=$\lozenge$,numberwithin=chapter]{gegenexmp}
\declaretheorem[name={Übungsaufgabe},style=style2,numberwithin=chapter]{uea}
\declaretheorem[name={Bemerkung},style=style1,qed=$\blacklozenge$,numberwithin=chapter]{rem}

% Vektor
\DeclareRobustCommand{\vect}[1]{\overrightarrow{#1}}   %Vektor
\DeclareRobustCommand{\vectHomog}[1]{{\overrightarrow{#1}}^{H}}   %Vektor in homog Darstellung
\DeclareRobustCommand{\matr}[1]{\bm{#1}} %Matrix
%\newcommand{\vect}[1]{\boldsymbol #1}
\newcommand{\skalar}[2]{\left \langle #1,#2 \right \rangle} %Skalarprodukt
\newcommand{\transp}[1]{#1^{\text{T}}}
\DeclareRobustCommand{\inv}[1]{\ensuremath{#1^{-1}}}

% Caption with defined width
\newcommand{\wcaption}[2]{%
   \begin{minipage}{#1}%
   \caption{#2}%
   \end{minipage}%
}

\newcommand*{\rechterWinkel}[3]{% #1 = point, #2 = start angle, #3 = radius
   \draw[shift={(#2:#3)}] (#1) arc[start angle=#2, delta angle=90, radius = #3];
   \fill[shift={(#2+45:#3/2)}] (#1) circle[radius=1.25\pgflinewidth];
}

\newcommand{\figureref}[1]{Abbildung \ref{#1}}%
\newcommand{\eqnref}[1]{Gleichung (\ref{#1})}%

% Command for margin text with usefull style
%\newcommand{\marginlabel}[1]{\mbox{}\marginline{\hspace{0pt}\footnotesize\sffamily #1}}%
\newcommand{\marginlabel}[1]{\marginnote{#1}}%

%\newcommand{\comment}[1]{\marginnote{#1}}%

% Enable space for figures that extent into the margin (right and/or leftside)
% Can be used inside a figure
% Note: sidecap defines a similar environment 'wide' !
\newenvironment{widespace}[2]{%
   \begin{list}{}{%
      \setlength{\topsep}{0pt}%
      \setlength{\leftmargin}{#1}%
      \setlength{\rightmargin}{#2}%
      \setlength{\listparindent}{\parindent}%
      \setlength{\itemindent}{\parskip}%
   }%
   \item[]%
}%
{%
   \end{list}%
}%

\newlength{\marginwidth}
\setlength{\marginwidth}{\marginparwidth}
\addtolength{\marginwidth}{\marginparsep}

%% Beispiel:
% \begin{figure}
% \begin{widespace}{-\marginwidth}{0pt}
%  \subfloat[Bergzebrastute]
%  {\includegraphics[width=0.45\linewidth]{../Bilder/Eingewoehnung2.jpg}}
%  \hspace*{1em}
%  \subfloat[Morro Moco]
%  {\includegraphics[width=0.45\linewidth]{../Bilder/bergzebra2.jpg}}
% \end{widespace}
% \end{figure}


% quantum optics - Latex Commands: Math **********************************
% ------------------------------------------------------------------------
% by: Matthias Pospiech
%%%%%%%%%%%%%%%%%%%%%%%%%%%%%%%%%%%%%%%%%%%%%%%%%%%%%%%%%%%%%%%%%%%%%%%%%%


% --| Math |-------------------------------------------------------

% -- Replacements --
\newcommand{\comp}{\ast}
%\renewcommand{\dagger}{+}

% -- Footnode --
\renewcommand{\thefootnote}{\arabic{footnote}}

% -- new commands --
\providecommand{\abs}[1]{\lvert#1\rvert}
\providecommand{\Abs}[1]{\left\lvert#1\right\rvert}
%\providecommand{\norm}[1]{\left\Vert#1\right\Vert}
\DeclareRobustCommand{\norm}[1]{\left\Vert#1\right\Vert}
\providecommand{\Trace}[1]{\ensuremath{\Tr\{\,#1\,\}}} % Trace /Spur
\providecommand{\KOS}[1]{\ensuremath{\mathit{#1}}} %Darstellung fuer Koordinatensysteme
\providecommand{\of}[1]{\ensuremath{\left(#1 \right)}} %parametrischer Zusammenhang einer Groesse
%

% -- differentials --
\renewcommand{\d}{\partial\mspace{2mu}} % partial diff
\newcommand{\td}{\,\mathrm{d}}	% total diff
\newcommand{\ddt}[1]{\frac{\td #1}{\td t}}

% -- Abbrevitations --
\renewcommand{\Re}{\text{Re}}			% Real value
\renewcommand{\Im}{\text{Im}}			% Real value
\newcommand{\complex}{\mathbb{C}} % Complex
\newcommand{\real}{\mathbb{R}}    % Real
%\newcommand{\R}{\real}						% Real
%\newcommand{\N}{\mathbb{N}}
%\newcommand{\Z}{\mathbb{Z}}
\DeclareRobustCommand{\set}[1]{\mathcal{#1}} %Darstellung von Mengen
\renewcommand{\L}{\set{L}}
\newcommand{\N}{\set{N}}
\newcommand{\R}{\set{R}}
\newcommand{\D}{\set{D}}

%
\newcommand{\Ham}{\mathcal{H}}    % Hamilton
\newcommand{\Prob}{\mathscr{P}}    
\newcommand{\unity}{\mathds{1}}   % 1-Vektor
\newcommand{\Fou}{\mathscr{F}}
%

\newcommand\gammab{\gamma_\bot}
\newcommand\gammap{\gamma_\parallel}
\newcommand\gammai{\gamma_\text{int}}
\newcommand\gammae{\gamma_\text{ext}}

% -- New Operators --
\DeclareMathOperator{\rot}{rot}
\DeclareMathOperator{\grad}{grad}
%\DeclareMathOperator{\div}{div}
\renewcommand{\div}{\text{div}\,}
\DeclareMathOperator{\Tr}{Tr}
\DeclareMathOperator{\const}{const}
\DeclareMathOperator{\Rang}{Rang}
\DeclareMathOperator{\Spur}{Spur}
\DeclareMathOperator{\Inter}{int}
\DeclareMathOperator*{\argmin}{arg\,min}
\DeclareMathOperator{\bd}{bd}
\DeclareMathOperator{\cl}{cl}
\DeclareMathOperator{\diag}{diag}
\DeclareMathOperator{\galois}{GF}
\DeclareMathOperator{\Cov}{Cov}
\DeclareMathOperator{\Var}{Var}
\DeclareMathOperator{\E}{E}
\DeclareMathOperator{\Span}{span}
\DeclareMathOperator{\Image}{im}
\DeclareMathOperator{\erf}{erf}
\DeclareMathOperator{\erfc}{erfc}

\newcommand{\vlaplace}[1][]{\mbox{\setlength{\unitlength}{0.1em}%
                            \begin{picture}(10,20)%
                              \put(3,2){\circle{4}}%
                              \put(3,4){\line(0,1){12}}%
                              \put(3,18){\circle*{4}}%
                              \put(10,7){#1}
                            \end{picture}%
                           }%
                     }%

\newcommand{\vLaplace}[1][]{\mbox{\setlength{\unitlength}{0.1em}%
                            \begin{picture}(10,20)%
                              \put(3,2){\circle*{4}}%
                              \put(3,4){\line(0,1){12}}%
                              \put(3,18){\circle{4}}%
                              \put(10,7){#1}
                            \end{picture}%
                           }%
                     }%

% -- new symbols --
%\newcommand{\laplace}{\Delta}
\newcommand{\dalembert}{\Box}

% -- new arrows --
\renewcommand{\leadsto}{\Longrightarrow}
\newcommand{\leftrightleadsto}{\Longleftrightarrow}


% -- Text subscripts--
\newcommand{\rel}{_\text{rel}}
%\newcommand{\st}{\text{st}}
%

% -- other --
\newcommand{\com}[2]{\underbrace{#1}_{\textrm{\scriptsize #2}}}
\newcommand{\with}[1]{\stackrel{\ref{#1}}{\Longrightarrow}}
%\newcommand{\unit}[1]{\,\textrm{#1}}

%\newcommand{\variance}[1]{\delta \mean{#1}^2}
\newcommand{\variance}[1]{(\Delta{#1})^2}
%\newcommand{\variance}[1]{\delta #1^2}

%Signum Funktion
\DeclareMathOperator\sgn{sgn}
% -- Physics --------------------------------
\newcommand\op[1]{{\hat{\mathrm{#1}}}}  % Operator

\newcommand\expect[1]{\ensuremath{\left\langle{#1}\right\rangle}} %
%
\newcommand{\mean}[1]{\ensuremath{\overline{#1}}} % mean value
%
\newcommand{\state}[1]{\ensuremath{\ket{#1}}}
%
\newcommand\commutator[2]{\ensuremath{\mathinner{%
    \mathopen[\,#1,#2\,\mathclose]}}}
\newcommand{\Commutator}[2]{\ensuremath{\left[\,#1,#2\,\right]}}
\newcommand{\bigcommutator}[2]{\ensuremath{\bigl[\,#1,#2\,\bigr]}}
\newcommand{\Bigcommutator}[2]{\ensuremath{\Bigl[\,#1,#2\,\Bigr]}}
%
\newcommand\poisson[2]{\mathinner{%
    \mathopen\{#1,#2\mathclose\}}}
%

% -- Layout --------------------------------

\newcommand*{\dashfill}{\leavevmode\cleaders\hbox{-}\hfill\kern0pt}

\newcommand*{\midhrulefill}{
\leavevmode
\cleaders\hbox to 1ex{\raisebox{.5ex}{\rule{1ex}{.4pt}}}\hfill\kern0pt
}

% -- Doppeltes Unterstreichen -------------
\newcommand\uli[2][]{
 \underline{#2\vphantom{#1}}
}

\newcommand\ulalign[2]{\uli[#2]{#1}&\uli[#1]{\;=#2}}
\newcommand\dulalign[2]{\underline{\uli[#2]{#1}}&\underline{\uli[#1]{\;=#2}}}


% -- SI-Einheiten -------------------------
\newcommand{\acrounit}[1]{
\acroextra{\makebox[18mm][l]{\si{#1}}}
}

% -- Leerseite
\newcommand\leerseite{\newpage\thispagestyle{empty}\hspace{1cm}\newpage}

% -- Anhangsinhaltsverzeichnis ------------
\makeatletter
% Die folgende Anweisung wird vermutlich irgendwann in scrlfile.sty eingebaut.
% Bis dahin ist es notwendig, sie selbst zu definieren, damit man innerhalb
% von \BeforeClosingMainAux \addtocontents verwenden kann:
\providecommand{\protected@immediate@write}[3]{%
  \begingroup
    \let\thepage\relax
    #2%
    \let\protect\@unexpandable@protect
    \edef\reserved@a{\immediate\write#1{#3}}%
    \reserved@a
  \endgroup
  \if@nobreak\ifvmode\nobreak\fi\fi
}
 
% Die folgende Umgebung wird verwendet, um innerhalb der toc-Datei einzelne
% Bereiche ein- und ausschalten zu können. In die toc-Datei wird die Umgebung
% dabei jeweils als \begin{tocconditional}{BEREICH}...\end{tocconditional}
% eingefügt.
\newenvironment*{tocconditional}[1]{%
  \expandafter\ifx\csname if@toccond@#1\expandafter\endcsname
                  \csname iftrue\endcsname
  \else
    \value{tocdepth}=-10000\relax
  \fi
  \typeout{tocdepth in `#1': \the\c@tocdepth}%
}{%
}
 
% Gleich nach dem Öffnen der toc-Datei beginnen wir den Haupt-Bereich "main":
\AtBeginDocument{%
  \addtocontents{toc}{\string\begin{tocconditional}{main}}
  \hypersetup{
    pdftitle = {\@title},
    pdfauthor = {\@author}
  }
}
% Und der letzte Bereich endet am Ende der toc-Datei.
\BeforeClosingMainAux{%
  \begingroup
    \let\protected@write\protected@immediate@write
    \addtocontents{toc}{\string\end{tocconditional}}%
  \endgroup
}
 
% Hier können nun neue Bereiche definiert ...
\newcommand*{\newtocconditional}[2][false]{%
  \expandafter\newif\csname if@toccond@#2\endcsname
  \csname @toccond@#2#1\endcsname
}
% Und ein- oder ausgeschaltet werden:
\newcommand*{\settocconditional}[2]{%
  \csname @toccond@#1#2\endcsname
}
 
% Neben dem Hauptbereich ...
\newtocconditional[true]{main}
% definieren wir noch einen Bereich für den Anhang.
\newtocconditional{appendix}
 
% Mit dem Anhang geben wir einerseits das Anhangsverzeichnis aus,
% andererseits beenden wir den aktuellen Bereich in der toc-Datei und beginnen
% den neuen Bereich "appendix". Damit im Haupt-Inhaltsverzeichnis ein Eintrag
% für das Anhangsverzeichnis erscheint, verwenden wir \addchap und zwar noch
% bevor der letzte Bereich geschlossen wird. Wenn wir es ganz sicher machen
% wollten, müssten wir die auskommentierten Zeilen noch aktivieren. So
% verlassen wir uns einfach darauf, dass vor dem appendix-Bereich der
% main-Bereich lag.
\g@addto@macro\appendix{%
%  \addtocontents{toc}{\string\end{tocconditional}^^J
%    \string\begin{tocconditional}{main}}%
  \addchap{Anhang}%
  \addtocontents{toc}{\string\end{tocconditional}
    \string\begin{tocconditional}{appendix}}%
  \appendixtableofcontents
}
 
% Jetzt definieren wir das Anhangsverzeichnis selbst als Alias für die
% toc-Datei. Dabei wird aber der Hauptbereich "main" deaktiviert und der
% Anhangsbereich "appendix" aktiviert.
\newcommand*{\appendixtableofcontents}{%
  \showtoc[{ %
    \aliastoc{\tocstyleTOC}{toc}%
    \settocconditional{main}{false}%
    \settocconditional{appendix}{true}%
  }]{toc}%
}
 
% Auch wenn man einen Anhang normalerweise nicht beenden kann, so ist es
% ggf. erwünscht, dass Literaturverzeichnis, Index etc. zwar nach den Kapiteln
% des Anhangs kommen, aber dem Hauptverzeichnis zugeordnet werden sollen. Also
% benötigen wir eine Anweisung, um in der toc-Datei den aktuellen Bereich zu
% beenden und wieder einen Hauptbereich einzuschalten:
\newcommand*{\postappendix}{%
  \addtocontents{toc}{\string\end{tocconditional}^^J%
      \string\begin{tocconditional}{main}}%
  \clearpage
}
 
\makeatother
\input{latex_base/general/macros/TableCommands}

%% Dokument Beginn
%% Gliederung:
%%		Deckblatt
%%		Inhaltsverzeichnis..................I
%%		Abkürzungsverzeichnis...............II
%%		Verwendete Formelm..................III
%%		Verwendete Indizies.................IV
%%		Abbildungsverzichnis................V
%%		(Tabellenverzeichnis................VI)
%%		1. Einleitung.......................1
%%		2. .................................3
%%		.
%%		.
%%		.
%%		Literatur- und Quellenverzeichnis...56
%%		Erklärung...........................60
%%		(Danksagung.........................61)
%%		Anhang..............................A-1

\begin{document}
% Deckblatt
\subject{Studienarbeit}
% \subject{Diplomarbeit \\ Universität <einfügen>}
% \title{<Titel einfügen>}
% \author{<Autor einfügen>}
% \date{<Datum einfügen>}
% \maketitle

% \begin{titlepage}
% 	\mbox{}\vspace{5\baselineskip}\\
% 	\sffamily\huge
% 	\centering
% 	<Titel einfügen>
% 	\vspace{3\baselineskip}\\
% 	\rmfamily\Large
% 	Diplomarbeit \\ Universität <einfügen>
% 	\vspace{2\baselineskip}\\
% 	\rmfamily\Large
% 	<Autor einfügen>
% 	\vspace{1\baselineskip}\\
% 	<Datum einfügen>
% \end{titlepage}


% \begin{titlepage}
% 	\sffamily\huge
% 	\centering
%	Diplomarbeit
%  	\vspace{3\baselineskip}\\
% 	\rmfamily\huge\bfseries
%	\centering
% 	Hier steht der Titel
% 	\vspace{8\baselineskip}\\
% 	\rmfamily\small
% 	Marius M\"uller \\
% \end{titlepage}

% Name des Verfassers
\author{Marius M\"uller}
\matrikelNr{3661272}
% Geburtsort
\geburtsort{Dresden}
% Geburtsdatum
\geburtsdatum{29. September 1989}
% Titel der Arbeit
\title{Modellierung technischer Systeme mit Hilfe homogener Koordinaten am Beispiel eines Motorradmodells}
% Angabe der Betreuer
\betreuer{Dipl.-Ing. Markus K\"obe (TU Dresden LKT)}
\betreuer{Dipl.-Ing. Robert Richter (TU Dresden ITVS)}
% Datum der Einreichung
\date{15.01.2017}
% Titelseite erstellen
\maketitleITVS


\chapter*{Bibliografischer Nachweis}
\thispagestyle{empty}
%\pagestyle{empty}
Marius M\"uller\\[2ex]
\textbf{Modellierung technischer Systeme mit Hilfe homogener Koordinaten am Beispiel eines Motorradmodells} \newline
Studienarbeit: 73 Seiten, 5 Abbildungen, 67 Literaturangaben\\
15.01.2017 \\
Technische Universit\"at Dresden \\
Fakult\"at Verkehrswissenschaften \glqq Friedrich List\grqq \\
Institut für Verkehrstelematik\\[2ex]
Autorenreferat:\\
Technische Systeme k\"onnen auf verschiedenste Arten und zu verschiedensten Zwecken modelliert werden. Bei mechanischen Systemen ist meist die Bewegung von Interesse. Die Aufteilung in mehrere Starrk\"orper ist ein \"ublicher Ansatz der Modellbildung. Ein derart erhaltenes Mehrk\"orpersystem muss so durch Parameter beschrieben werden, dass die Lage aller Teilk\"orper bestimmt, und damit die Bewegung des Gesamtsystems beschrieben werden kann. Die Wahl von Parametern nach der Methode der nat\"urlichen Koordinaten wird erl\"autert. Die Bewegungsbeschreibung geschieht \"ublicherweise mit Vektoren, welche als Elemente des $\set{R}^{3}$ aufgefasst werden. Die mathematischen Grundlagen dieser Beschreibungsform werden dargelegt. Die Repr\"asentation des Ortes als Tripel erlaubt keine Darstellung der Starrk\"orperbewegung in Form einer linearen Abbildung. Dieses Problem wird durch die Einf\"uhrung homogener Koordinaten gel\"ost. Mit Hinblick auf diese Variante der Koordinatendarstellung werden Grundprinzipien der Kinematik hergeleitet und es wird auf die L\"osung von Bewegungsgleichungen eingegangen. Die Verwendbarkeit dieser Darstellung wird anhand ausgew\"ahlter Modellgleichungen, mit denen ein Motorrad beschrieben werden kann, gezeigt.



\newpage
\thispagestyle{empty}
\null\vfill
\begin{center}
	Bitte ersetzen Sie diese Seite vor dem Binden mit der Aufgabenstellung.
\end{center}
\vfill

%% Selbstständigkeitserklärung
% Ort der Selbstständigkeitserklärung (Standard: Dresden)
\selbstort{Dresden}
% Datum der Selbstständigkeitserklärung (Standard: aktuelles Systemdatum)
\selbstdatum{15. Januar 2017}
\selbststaendigkeitserklaerung
%% Kurzfassung / Abstract
%\kurzfassung{An dieser Stelle fügen Sie bitte eine deutsche Kurzfassung ein.}{Please insert the English abstract here.}
%\thispagestyle{empty}
%\chapter*{Erklärung der Selbstständigkeit}
%\thispagestyle{empty}
%Hiermit versichere ich, die vorliegende Arbeit selbstständig verfasst und keine anderen als die angegebenen Quellen und Hilfsmittel benutzt sowie die Zitate deutlich kenntlich gemacht zu haben.
%\vspace{4\baselineskip}\\
%<Ort einfügen>, den <Datum einfügen> \hfill <Autor einfügen>
%\vspace{4\baselineskip}\\
%\clearpage
%\mbox{}\thispagestyle{empty}

\cleardoublepage
\frontmatter
\cleardoublepage
% Inhaltsverzeichnis in den PDF-Links eintragen
\tableofcontents
\cleardoublepage
% Abkuerzungsverzeichnis
\chapter*{Abkürzungsverzeichnis}
\addcontentsline{toc}{chapter}{Abkürzungsverzeichnis}
\begin{acronym}[LabVIEW] %<--in Klammern das laengste Wort
	\acro{FDM}{Finite Differenzen Methode}
\end{acronym}
\cleardoublepage
% Verzeichnis der Formelzeichen
\chapter*{Verzeichnis der verwendeten Formelzeichen}
\addcontentsline{toc}{chapter}{Verzeichnis der verwendeten Formelzeichen}
\begin{acronym}[LabVIEW] %<--in Klammern das laengste Wort
	\acro{alpha}[\ensuremath{\alpha}]{ \acrounit{\meter^2\per\second}Temperaturleitfähigkeit}
	\acro{rho}[\ensuremath{\rho}]{ \acrounit{\kilo\gram\per\meter^3}Dichte}
	\acro{c}[\ensuremath{c}]{ \acrounit{\joule\per\kilo\gram\per\degreeCelsius}spezifische Wärmekapazität}	
	\acro{k}[\ensuremath{k}]{ \acrounit{\watt\per\meter\per\degreeCelsius}thermische Leitfähigkeit}	
	\acro{L}[\ensuremath{L}]{ \acrounit{\joule\per\kilo\gram}latente Wärme}	
\end{acronym}
\cleardoublepage
% Verzeichnis der verwendeten Indizes
\chapter*{Verzeichnis der verwendeten Indizes}
%\addcontentsline{toc}{chapter}{Verzeichnis der verwendeten Indizes}
\begin{acronym}[LabVIEW] %<--in Klammern das laengste Wort
	\acro{l}{liquid/flüssig}
	\acro{s}{solid/fest}
	\acro{i}{interface/Grenzschicht}
	\acro{m}{melting point/Schmelzpunkt}
	\acro{U}{Unterseite}
	\acro{O}{Oberseite}
\end{acronym}
\cleardoublepage
% Verzeichnis der verwendeten Symbole
\chapter*{Symbolverzeichnis}
%\addcontentsline{toc}{chapter}{Symbolverzeichnis}
\begin{acronym}[Bedeutungen] %<--in Klammern das laengste Wort
	\acro{Notation}{\textbf{Bedeutung}}
	\acro{2-norm}[\ensuremath{\norm{ \cdot }}]{euklidische Norm}
	\acro{vector}[\ensuremath{\vect{a}}]{Vektor}
	\acro{matrix}[\ensuremath{\matr{A}}]{Matrix}
	\acro{skalarProd}[\ensuremath{\skalar{\cdot}{\cdot}}]{Skalarprodukt}
	\acro{vectorProd}[\ensuremath{{\cdot}\times{\cdot}}]{Vektorprodukt}
%	\acro{transp}[\ensuremath{ \transp{\left( \matr{A} \right)}}]{Transponierte Matrix A}
%	\acro{rotMatr}[\ensuremath{\matr{R}}]{Rotationsmatrix}
%	\acro{transfMatr}[\ensuremath{\matr{T}}]{Transformationsmatrix mit Rotation und Translation}
%	\acro{virtVersch}[\ensuremath{\delta \vect{r}}]{virtuelle Verschiebung}
%	\acro{virtRot}[\ensuremath{\delta \vect{\phi}}]{virtuelle Rotation}
%	\acro{virtW}[\ensuremath{\delta W}]{virtuelle Arbeit}
%	\acro{genF}[\ensuremath{\vect{Q}}]{generalisierte Kraft}
\end{acronym}
\cleardoublepage
% Abbildungs- und Tabellenverzeichnis
\listoffigures
\cleardoublepage
\listoftables
\cleardoublepage

% Hauptteil
\mainmatter
\chapter{Stand der Technik}\label{ch:standDerTechnik}
\section{Starrk\"orperbewegung}\label{sec:starrkoerperbewegung}
Die Bewegung eines Punktes $p$ im euklidischen Raum wird durch die Angabe seiner Position in Bezug zu einem inertialen Koordinatensystem $\KOS{I}$ zu jedem Zeitpunkt $\acs{t}$ eindeutig beschrieben. Das inertiale Koordinatensystem $\KOS{I}\in \R^{3}$ habe die Basisvektoren $\vect{e}_1,\vect{e}_2,\vect{e}_3$. F\"ur die Basisvektoren gelte: \begin{align*}
\skalar{\vect{e}_i}{\vect{e}_j}&=
\begin{cases} 
1, \text{ f\"ur } i=j \\
0, \text{ f\"ur } i\neq j \end{cases} 
\intertext{und weiterhin}
\vect{e}_1 \times \vect{e}_2 &= \vect{e}_3
\end{align*}
Die Basisvektoren von $\KOS{I}$ beschreiben damit ein orthonormales Rechtssystem (siehe beispielsweise \cite[S. 80]{Papula2014}). \newline
Die Position des Punktes $p$ sei durch das Tripel $\left( x, y, z \right) \in \R^{3}$ gegeben. Die Trajektorie von $p$ kann dann durch die parametrisierte Bahn $p\of{t}=\left(x\of{t}, y\of{t}, z\of{t}\right) \in \R^{3} $ beschrieben werden. Da nicht die Bewegung von einzelnen Punkten, sondern die Bewegung eines Starrk\"orpers beschrieben werden soll, wird der Begriff Starrk\"orper definiert.

\begin{defn}[Starrk\"orper] Ein Starrk\"orper ist dadurch gekennzeichnet, dass die Distanz zweier beliebiger Punkte $p, q$ unabh\"angig von der Bewegung des K\"orpers, immer konstant bleibt. Die anf\"angliche Position des Punktes $p$ sei beschrieben durch $p\of{0}$. Die Position nach einer beliebigen Zeit $\acs{t}$ (und einer beliebigen Bewegung) sei beschrieben durch $p\of{t}$. Die Nomenklatur gilt f\"ur den Punkt $q$ analog. F\"ur einen Starrk\"orper wird dann gefordert: \begin{align*}
\norm{p\of{t} - q\of{t}}&=\norm{p\of{0}-q\of{0}} = \text{konstant}
\end{align*}
\end{defn}

  \subsection{Koordinatensysteme}
	





% Anhang (Bibliographie darf im deutschen nicht in den Anhang!)
%\bibliography{bib/quellen}

\nocite{*}%alle Elemente im Literaturverzeichnis werden aufgelistet, auch die nicht-zitierten
\printbibliography[title={Literaturverzeichnis}]
\pagenumbering{Roman} %neue Seitenzahlen
\stepcounter{chapter}
\addcontentsline{toc}{chapter}{\protect\numberline{\thechapter}{Literaturverzeichnis}}
\cleardoublepage

% Anhang
\backmatter
\appendix
\chapter{Anhang mit Sachen}
\label{cha:anhang_uebungen}


\IfDefined{printindex}{\printindex}
\IfDefined{printnomenclature}{\printnomenclature}
\cleardoublepage
\leerseite{}
\cleardoublepage


\end{document}

